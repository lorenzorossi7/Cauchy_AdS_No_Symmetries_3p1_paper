\documentclass[12pt, twoside]{article}

%\usepackage[OT4]{fontenc}
%\usepackage[cp1250]{inputenc}
%\usepackage[polish]{babel}
%\usepackage{latin2}
\usepackage[top=1.5cm, bottom=2cm, left=2.0cm, right=2.0cm]{geometry}
\usepackage[T1]{fontenc}
\usepackage[utf8]{inputenc}
%\usepackage{polski}
\usepackage{graphicx} 
\usepackage{amsfonts}
\usepackage[hidelinks]{hyperref}
%\usepackage{amssymb}

%\pagestyle{empty}

\begin{document}

\begin{center}
\textbf{
Report on the manuscript\\
\textit{Cauchy Evolution of Asymptotically Global AdS Spacetimes with No
Symmetries}
\\
by Hans Bantilan, Pau Figueras, and Lorenzo Rossi}
\end{center}
\vspace{0.2cm}

\noindent

The authors discuss the Cauchy evolutions of asymptotically global AdS
spacetimes with no imposed symmetries, employing a numerical scheme based on the
generalized harmonic form of the Einstein equations. This is an excellent work
and I have no doubts that it merits publication in \textit{Physical Review D}.
Still, I have one suggestion that in my opinion might clarify the contnent of
Sections II and III: on page 3 the authors define the compatification scale
$\ell$ and set this scale to one but at this stage it is not clear from the text
if this compactification scale is related to the AdS radius $L$ (see (II.4)) or
not. The information that the AdS radis $L$ is also set to one (thus, in fact,
$\ell=L$) can be found not earlier then in Sec. VI. In my opinion this fact
should be explicitely highlited already in Sec. II because it probably has
signifcant consequences. In particular, I would expect the formulas
(III.21-III.24) (and probably also (III.11-III.20)) to be valid only under this
simplifying assumption (i.e. $\ell=L$), since in this case the metric (II.6)
simplifies significantly (it takes a diagonal form), and (III.21-III.24) can be
indeed easily obtained. On the other hand I was unable to reproduce
(III.21-III.24) for the general $\ell \neq L$ case. Thus, I would like to know
whether it is my failure or the authors do assume $\ell=L$ implicitly in their
derivations. Similarily, as far as I understand, (III.8) is simply $\sim \hat
g^{\rho\sigma} \partial_{\rho} \partial_{\sigma} \bar g_{(1)\mu\nu}$ thus, for a
general (off-diagonal) form of (II.6), I would expect mixed derivatives to be
present in (III.8). Similarily, I would imagine that coding the formulas
containing the inverse of $g_{\mu\nu} = \hat g_{\mu\nu} + h_{\mu\nu}$ would be a
nightmare for the general case $\ell \neq L$. If I am correct about the
assumption $\ell=L$ being used, I would urge the authors to state it explicitly
and to simplify the form of (II.6) accordingly.
\\

\noindent
Apart from the comment above I have just 2 minor remarks:
\begin{enumerate}
\item It would be nice to have correct diacritic in the names of the authors
citetd in the refernces. In particular, in the case of [41] it can be achieved
with Jalmuzna $\rightarrow$ \verb+Ja\l mu\.zna+ (to produce Ja\l mu\.zna)
\item I find the sentence \textit{we see that the scalar field starts
propagating towards the AdS boundary, but a significant portion of it is
attracted back towards the origin} (p.12/13) a bit misleading: the authors use
time-symmetric intial data thus they naturally contain both outgoing and ingoing
components. 

\end{enumerate} 

To summarize, the manuscript reports on excellent research project and I am
looking forward to seeing it published in PRD.
\end{document}