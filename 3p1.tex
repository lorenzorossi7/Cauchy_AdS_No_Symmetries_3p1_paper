\documentclass[12pt]{iopart} %change into article
\usepackage{iopams}  
\usepackage{graphicx}
\usepackage{subfig}
\usepackage{xcolor}
\setlength\parindent{0pt}
\setlength{\parskip}{13pt} %found this single-line pt value using \the\baselineskip
\begin{document}

\title[]{Cauchy Evolution of Asymptotically AdS Spacetimes with No Symmetries}

\author{Hans Bantilan$^1$, Pau Figueras$^1$, Lorenzo Rossi$^1$}
\address{$^1$ School of Mathematical Sciences, Queen Mary University of
  London, \\ Mile End Road, London E1 4NS, United Kingdom}
\ead{l.rossi@qmul.ac.uk}

\begin{abstract}
We present the first proof-of-principle Cauchy evolutions of asymptotically AdS spacetimes with no symmetries.
We use a numerical scheme that is based on the generalized harmonic form of the Einstein equations expressed in Cartesian coordinates.
For this first study, we perform 3+1 simulations of four dimensional spacetimes with a negative cosmological constant, using initial data sourced by a massless scalar field.
We demonstrate stable and convergent Cauchy evolution of horizonless data, as well as data that undergoes gravitational collapse.
The main difficulty in removing all symmetry restrictions in this setting is to find a generalized harmonic gauge choice that is consistent with the AdS boundary conditions.
We explicitly write down the gauge choice that achieves this in terms of a specific choice of generalized harmonic source functions.
These preliminary results are a direct precursor to important specialized studies of AdS dynamics with no symmetry assumptions, most notably of gravitational collapse, and of the superradiant instability.
\end{abstract}


\noindent{Keywords}: AdS, generalized harmonic, 3+1, superradiance, gravitational collapse

\textcolor{blue}{This colour is for something we should discuss whether to add or not.}

% Uncomment for Submitted to journal title message
%\submitto{\JPA}
%
% Uncomment if a separate title page is required
%\maketitle
% 
% For two-column output uncomment the next line and choose [10pt] rather than [12pt] in the \documentclass declaration
%\ioptwocol




\section{Introduction}

In an unprecedented way, the simulation of asymptotically anti-de Sitter (AdS) spacetimes has opened up the field of numerical relativity to the study of phenomena in areas beyond General Relativity (GR). 
At the heart of this push to understand AdS is a deep mathematical connection between gravity in AdS to certain conformal field theories (CFT), now known as the AdS/CFT correspondence~\cite{Maldacena:1997re,Gubser:1998bc,Witten:1998qj}. 
Through this connection, the study of AdS spacetimes has become immediately relevant to fundamental questions in many areas in physics, such as fluid dynamics [cite], relativistic heavy ion collisions [cite], and superconductivity [cite].
See, for example, [cite] for excellent reviews. 

The reason the study of AdS is crucial for our understanding of these phenomena is that AdS/CFT provides an important --and in most cases the only-- window into the real-time dynamics of quantum field theories far from equilibrium. 
The dynamical far-from-equilibrium regime is precisely the one that is least explored and understood, and the one that has the best chance of making contact with experiment.
Within GR, the dynamics of AdS provides a detailed look at how the Einstein equations behave in the fully non-linear regime and with non-trivial boundary conditions.
This has led to important progress in our understanding fundamental phenomena in gravity, such as gravitational collapse, black hole collisions, and superradiance. 

Our current understanding of AdS, however, remains limited. 
This is due to several reasons.
First, evolution in AdS is notoriously hard, in part because it is an initial boundary value problem whose systematic study is still in its infancy. 
Cauchy evolution in AdS requires data to be prescribed not only at an initial space-like hypersurface, but also at spatial and null infinity which constitute the time-like boundary of an asymptotically AdS spacetime.
Second, the most interesting phenomena involve spacetimes that have very little or no symmetry, making these evolutions beyond the reach of most numerical codes. 
Third, for many of these phenomena, there is a variety of physical scales that must be adequately resolved to capture the correct physics.

The main purpose of this article is to present the first proof-of-principle Cauchy evolution of asymptotically AdS spacetimes that has been achieved with no symmetry assumptions.
The results presented here are based on a code that solves the Einstein equations for asymptotically AdS spacetimes with a generalized harmonic evolution scheme, with adaptive mesh refinement capabilities. 
This code is the next step in an ongoing program initiated in \cite{Bantilan:2017kok} that uses Cartesian coordinates in global AdS, now enabled with a breakthrough set of generalized harmonic source functions that is crucial in successfully stabilizing evolution in AdS with no symmetries.

This article is organized as follows. 
In Section~\ref{sec:setup} we describe the setup, and in Section~\ref{sec:numerical_scheme} we outline the generalized harmonic scheme that we use in our simulations.
In Section~\ref{sec:gauge_choice} we explicitly write down our gauge choice in terms of a specific choice of generalized harmonic source functions that stabilizes these evolutions with no symmetry.
Section~\ref{sec:results} contains preliminary results from this new code, and we conclude with a discussion in Section~\ref{sec:Discussion}.



\section{Setup}\label{sec:setup}

\subsection{Anti-de Sitter Spacetime}\label{subsec:AdS}
The dynamics of gravity with a cosmological constant $\Lambda$ in four dimensions coupled to a real massless scalar field $\varphi$ can be described by the following action
\begin{equation}\label{eqn:action}
S = \int d^4 x \sqrt{-g} \left( \frac{1}{16\pi} \left( R - 2\Lambda \right) - g^{\alpha\beta} \partial_\alpha \varphi \partial_\beta \varphi \right),
\end{equation}
where $R$ is the Ricci scalar of the metric $g_{\alpha\beta}$ with determinant $g$.
Here, we use geometric units where Newton's constant $G$ and the speed of light $c$ are set to 1.
The variation of the action \eref{eqn:action} with respect to $g_{\alpha\beta}$ and $\varphi$ gives the equations of motion
\begin{eqnarray}\label{eqn:eoms1}
R_{\alpha\beta} - \frac{1}{2} R g_{\alpha\beta} + \Lambda g_{\alpha\beta} = 8\pi \left( \partial_\alpha \varphi \partial_\beta \varphi - g_{\alpha\beta} \frac{1}{2} g^{\gamma\delta} \partial_{\gamma} \varphi \partial_{\delta} \varphi \right),
\end{eqnarray}
\begin{equation}\label{eqn:eoms2}
g^{\alpha\beta} \nabla_{\alpha} \nabla_{\beta} \varphi = 0.
\end{equation}

We aim to solve this set of equations using the Generalized Harmonic formulation (see~\ref{sec:GHfor} for the form of the evolution equations \eref{eqn:eoms1},\eref{eqn:eoms2} in this framework and CITE PAPERS for more details about the theoretical aspects of the formulation), whose solution is given (in any set of coordinates) by the metric $g_{\alpha\beta}$, the scalar field $\varphi$ and a gauge choice of source functions $H_\alpha$. 

The metric of AdS$_4$ is the maximally symmetric vacuum (i.e. $\varphi=0$) solution of \eref{eqn:eoms1},\eref{eqn:eoms2} in four dimensions.
In terms of global coordinates $(t,r,\theta,\phi)\in(-\infty,+\infty)\times(0,+\infty)\times[0,\pi)\times[0,2\pi)$ that cover the whole spacetime, the metric of AdS$_4$ can be expressed as
\begin{equation}\label{eqn:ads4}
\hat{g}= \left( -(1+r^2/L^2) dt^2 + (1+r^2/L^2)^{-1} dr^2 +r^2 d{\Omega_2}^2 \right), \nonumber
\end{equation}
with a characteristic length scale $L$ that is related to the cosmological constant by $\Lambda = - 3/L^2$, and $d{\Omega_2}^2 = d\theta^2 + \sin^2\theta d\phi^2$ is the metric of the round 2-sphere. The crucial feature of this spacetime is the presence of a \emph{time-like} boundary at $r \rightarrow +\infty$, which makes stable evolution of initial data possible only if boundary conditions are imposed on the evolved fields. In other words, any Cauchy problem is an initial-boundary value problem.

First, we compactify $r=2\rho/(1-\rho^2/\ell^2)$ so that the AdS boundary at $r \rightarrow +\infty$ is at a finite value of the new radial coordinate, $\rho=\ell$.
We hereafter set $\ell=1$ without loss of generality, so that the AdS boundary is $\rho=1$. In this way, we obtain (compactified) spherical coordinates $x^\alpha=(t,\rho,\theta,\phi)$.
Defining a convenient function $\hat{f}(\rho) = (1-\rho^2)^2+4\rho^2/L^2$, the metric of AdS$_4$ in this set of coordinates reads
\begin{equation}\label{eqn:ads4_compact}
\hat{g}_{\alpha\beta}dx^{\alpha}dx^{\beta} = \frac{1}{(1-\rho^2)^2} \left( -\hat{f}(\rho) dt^2 + 4(1+\rho^2)^2 \hat{f}(\rho)^{-1} d\rho^2 + 4\rho^2 d{\Omega_2}^2 \right). \nonumber
\end{equation}

Second, we define Cartesian coordinates $x^\mu=(t,x,y,z)$ by $x=\rho\cos\theta$, $y=\rho\sin\theta\cos\phi$, $z=\rho\sin\theta\sin\phi$ where $\rho=\sqrt{x^2+y^2+z^2}$, in order to bypass the severe restriction on time step size at the origin $\rho=0$ of a grid in polar coordinates. 
This brings the metric of AdS$_4$ into its final form
\begin{eqnarray}\label{eqn:ads4_final}
\hat{g}_{\mu\nu}dx^{\mu}dx^{\nu}=&\frac{1}{\left(1-\rho^2\right)^2 }\left( -dt^2 \hat{f}(\rho) +4\rho^{-2}\hat{f}(\rho)^{-1} \left(1+\rho^2\right)^2 (x dx + y dy + z dz)^2 \right. \nonumber \\
&+\frac{4}{\rho^2} \left[\left(-2 x y\right) dx dy + \left(- 2 y z\right) dy dz + \left(- 2 x z\right) dx dz \right. \nonumber \\
&\left. \left. + \left(y^2+z^2\right) dx^2 + \left(x^2+z^2\right) dy^2 + \left(x^2+y^2\right) dz^2 \right] \right),
\end{eqnarray}

\subsection{Asymptotically Anti-de Sitter Spacetimes}\label{subsec:asyAdS}

We will be interested in the Cauchy evolution of asymptotically AdS spacetimes. In this Section we see how they have been defined in the literature and we find the boundary conditions that such spacetimes must satisfy.

%Their definition has been given in different ways in the literature, so we will have to state the one relevant for this study. 
%In [CITE Henneaux-Teitelbom], the authors implicitly assume that these spacetimes $(M,g)$ allow a definition of spatial infinity $\mathcal{I}$ (via conformal compactification) with the same topology as the AdS boundary, i.e. $\mathbb{R}\times S^2$.

% topology and conformal boundary metric given by a conformal transformation of the pure AdS metric $\hat{g}$, which means in particular that $\mathcal{I}$ is a time-like boundary of the conformally compactified spacetime.

%spatial infinity of pure AdS, i.e. $\mathbb{R}\times S^2$ topology and conformal boundary metric given by a conformal transformation of the pure AdS metric $\hat{g}$, which means in particular that $\mathcal{I}$ is a time-like boundary of the conformally compactified spacetime. Then, they proceed on defining asymptotically AdS spacetimes by requiring that, for any set of global coordinates $x^\alpha$, the deviation of the full metric $g_{\alpha\beta}$ from the pure AdS metric $\hat{g}_{\alpha\beta}$, given by $h_{\alpha\beta}=g_{\alpha\beta}-\hat{g}_{\alpha\beta}$, satisfies three conditions

Let us start from the definition presented in [CITE Henneaux-Teitelbom]. The authors implicitly assume that these spacetimes $(M,g)$ allow a definition of spatial infinity $\mathcal{I}$ (via conformal compactification). Then, they proceed on defining asymptotically AdS spacetimes by requiring that, for any set of global coordinates $x^\alpha$, the deviation of the full metric $g_{\alpha\beta}$ from the pure AdS metric $\hat{g}_{\alpha\beta}$, given by $h_{\alpha\beta}=g_{\alpha\beta}-\hat{g}_{\alpha\beta}$, satisfies three conditions
\begin{enumerate}
\item it is consistent with the asymptotic decay of the Kerr-AdS metric near $\mathcal{I}$ in that set of coordinates;
\item its fall-off near $\mathcal{I}$ is invariant under the global AdS symmetry group $O(3,2)$ i.e.
\begin{equation}\label{eqn:asyKilleq}
(\mathcal{L}_X h)_{\alpha\beta}=\mathcal{O}(h_{\alpha\beta})\;\;\;\; \textrm{near spatial infinity $\mathcal{I}$},
\end{equation}
for any $X$ generator of $O(3,2)$;
\item the surface integral charges associated with the generators of $O(3,2)$ are finite.
\end{enumerate}
More precisely, it is shown in [CITE Henneaux-Teitelbom] that the explicit fall-off satisfying (i) and (ii) automatically implies (iii).
Notice that (ii) implies that the full metric $g_{\alpha\beta}$ approaches the pure AdS metric $\hat{g}_{\alpha\beta}$ near $\mathcal{I}$. An important consequence is that $\mathcal{I}$ has the same conformal structure as the pure AdS boundary (thus we will still refer to it as AdS boundary), i.e. $\mathbb{R}\times S^2$ topology and conformal metric given by a conformal transformation of the purely AdS metric $\hat{g}_{\alpha\beta}$ at the boundary (e.g. the metric of the Einstein Static Universe), which implies that $\mathcal{I}$ is a time-like boundary of the conformally compactified spacetime. Moreover, given a set of coordinates $x^\alpha$ in which the pure AdS metric components are $\hat{g}_{\alpha\beta}$, we will denote by $x^\alpha$ all sets of coordinates in which the full metric components $g_{\alpha\beta}$ approach the pure AdS metric components in the form $\hat{g}_{\alpha\beta}$. For example, we will denote any set of coordinates in which the metric $g$ asymptotes to $\hat{g}$ in the form \eref{eqn:ads4_compact} by $(t,\rho,\theta,\phi)$ and we will refer to them as as spherical coordinates (although they should be regarded as asymptotically spherical coordinates, since they are well-defined only near the boundary $\mathcal{I}$). Similarly, we will denote any set of coordinates in which $g$ asymptotes to $\hat{g}$ in the form \eref{eqn:ads4_final} by $(t,x,y,z)$ and we will refer to them as Cartesian coordinates (although they should be regarded as asymptotically Cartesian coordinates).
Furthermore, \eref{eqn:asyKilleq} is explicitly solved in [CITE [1201.2132]] (this is done for any spacetime dimensions but we focus here on the 4 dimensional case) in (asymptotically) $(t,r,\theta,\phi)$ coordinates for a solution ansatz of the form $h_{\alpha\beta}\sim r^{p_{\alpha\beta}}$ and the result is the same as the one in [CITE Henneaux-Teitelbom], which shows that requiring (ii) is sufficient to obtain the fall-off near spatial infinity that satisfies also (i) and (iii).
In addition, given that in this work we only look at spacetimes that evolve according to the laws of General Relativity, we can restrict this definition to spacetimes that satisfy the Einstein equations.

Using the solution of \eref{eqn:asyKilleq} found in [cite 1201.2132], we obtain the boundary conditions in spherical coordinates:
\begin{eqnarray}
\label{eq:sphbounconh}
h_{\rho\alpha}&=f_{\rho\alpha}(t,\theta,\phi)(1-\rho)^2+\mathcal{O}((1-\rho)^3) \;\; \textrm{ if $\alpha\neq\rho$}, \\ \nonumber
h_{\alpha\beta}&=f_{\alpha\beta}(t,\theta,\phi)(1-\rho)+\mathcal{O}((1-\rho)^{2}) \;\; \textrm{ otherwise},
\end{eqnarray}
(where $m,n=t,\theta,\phi$) for arbitrary functions $f_{\alpha\beta}(t,\theta,\phi)$. Furthermore, we need to impose a boundary condition on the massless scalar field $\varphi$ that preserves the asymptotics \eref{eq:sphbounconh} during evolution (as verified in CITE [cite 1201.2132]):
\begin{equation}\label{eq:sphbounconphi}
\varphi=b(t,\theta,\phi)(1-\rho)^3+\mathcal{O}((1-\rho)^4)
\end{equation}
for arbitrary $b(t,\theta,\phi)$.

In Cartesian coordinates, these read
\begin{eqnarray}
\label{eq:carbouncondh}
h_{\mu\nu}&=&f_{\mu\nu}(t,x,y,z)|_{\rho(x,y,z)=1}(1-\rho(x,y,z))+\mathcal{O}((1-\rho(x,y,z))^{2}), \\
\label{eq:carbouncondphi}
\varphi&=&c(t,x,y,z)|_{\rho(x,y,z)=1}(1-\rho(x,y,z))^3+\mathcal{O}((1-\rho(x,y,z))^{4}), 
\end{eqnarray}
for arbitrary $f_{\mu\nu}$ and $c$.

The boundary conditions on the source functions in a given set of coordinates, involved in the Generalized Harmonic Formulation employed in this study (CITE), can be deduced from the near-boundary behaviour of the full metric $g$ through the definition 
\begin{equation}\label{eq:defsoufunsph}
H^\alpha \equiv \Box x^\alpha = \frac{1}{\sqrt{-g}}\partial_\beta (\sqrt{-g}g^{\beta\gamma}x^\alpha_{\;\;,\gamma})=\frac{1}{\sqrt{-g}}\partial_\beta (\sqrt{-g}g^{\beta\alpha})
\end{equation}
in spherical coordinates, and similarly in Cartesian coordinates. 
In spherical coordinates, denoting the pure AdS values by $\hat{H}_\alpha$, \eref{eq:sphbounconh} implies,
\begin{eqnarray}\label{eq:sphbouncondsoufunc}
H_\alpha&=&\hat{H}_\alpha+f_\alpha(t,\theta,\phi)(1-\rho)^3+\mathcal{O}((1-\rho)^4) \;\; \textrm{ if $\alpha\neq\rho$} \\ \nonumber
H_\rho&=&\hat{H}_\rho+f_\rho(t,\theta,\phi)(1-\rho)^2+\mathcal{O}((1-\rho)^3)
\end{eqnarray}
for arbitrary $f_\alpha$.
In Cartesian coordinates,  denoting the pure AdS values by $\hat{H}_\mu$, \eref{eq:carbouncondh} implies,
\begin{equation}\label{eq:carbouncondsoufun}
H_\mu=\hat{H}_\mu+f_\mu(t,x,y,z)|_{\rho(x,y,z)=1}(1-\rho(x,y,z))^2+\mathcal{O}((1-\rho(x,y,z))^3)
\end{equation}
for arbitrary $f_\mu$.


%Moreover, given a coordinate system $x^\alpha$ in which the pure AdS metric is known (e.g. $(t,r,\theta,\phi)$), this fact allows us to use the same notation for all coordinate systems in which the full metric approaches $\hat{g}$ in the form given by 


%has the same conformal structure of the pure AdS boundary, which is also the conformal structure of the AdS boundary, i.e. $\mathbb{R}\times S^2$ topology and conformal metric given by a conformal transformation of the purely AdS metric $\hat{g}_{\alpha\beta}$. Given these requirements, it is evident 
%It is shown in [CITE Henneaux-Teitelbom] that the explicit fall-off satisfying (i) and (ii) automatically implies (iii). Furthermore, \eref{eqn:asyKilleq} is explicitly solved in [CITE [1201.2132]] (this is done for any spacetime dimensions but we focus here on the 4 dimensional case) in $(t,r,\theta,\phi)$ coordinates for a solution ansatz of the form $h_{\alpha\beta}\sim r^{p_{\alpha\beta}}$ and the result is the same as the one in [CITE Henneaux-Teitelbom], which shows that requiring (ii) is sufficient to obtain the fall-off near spatial infinity that satisfies also (i) and (iii).

It is now interesting, as well as relevant for what follows, to discuss a second definition of asymptotically AdS spacetimes given in the literature. We start by defining a spacetime $(M,g)$  that is only locally asymptotically AdS, as a spacetime that admits a conformal compactification (which allows to define a boundary) and satisfies the Einstein equations \eref{eqn:eoms1}, without making any assumption, for example, on the topology of the boundary. The Fefferman-Graham (FG) theorem [cite C. Fefferman and C. Robin Graham, in Elie Cartan et les Mathematiques d’aujourdhui
(Asterisque, 1985) 95] states that for this type of spacetimes in 4 dimensions, we can always find a coordinate system $\bar{x}^\alpha=(\bar{t},\bar{z},\bar{\theta},\bar{\phi})$ in a neighbour of the boundary for which the boundary is at $\bar{z}=0$ and the metric can be written in the form
\begin{equation}
\label{eqn:FGmetric}
g=\frac{L^2}{\bar{z}^2}(d\bar{z}^2+g_{ij}(\bar{z},\bar{x})d\bar{x}^id\bar{x}^j),
\end{equation}
where 
\begin{equation}
\label{eqn:FGbdymetric}
g_{ij}(\bar{z},\bar{x})=g_{(0)ij}(\bar{x})+g_{(2)ij}(\bar{x})\bar{z}^2+\mathcal{O}(\bar{z}^3),
\end{equation}
in which the near-boundary (i.e. about $\bar{z}=0$) expansion of the Einstein equations completely determines the coefficient $g_{(2)ij}(\bar{x})$ in terms of $g_{(0)ij}(\bar{x})$, therefore the dynamics that makes this spacetime differ from pure AdS appears at order $\bar{z}^3$. If we make the further requirement that the topology of the boundary is that of pure AdS, i.e. $\mathbb{R}\times S^2$, the spacetime becomes globally asymptotically AdS and the definition becomes equivalent to the previous one. The FG form of the metric immediately provides the near-boundary behaviour of the metric and it shows that coordinates can be defined so that the ``radial''-``radial'' component of any asymptotically AdS metric is 1 and ``radial''-``non-radial'' components are 0 in a neighbourhood of the AdS boundary.

We conclude by showing an explicit example of how FG coordinates can be found in the case of $AdS_4$, setting $L=1$ for simplicity. Notice that, from  \eref{eq:sphbounconh}, we know all the metric components $g_{\alpha\beta}$ only up to and including $\mathcal{O}(1-\rho)$ so we can expect to find the FG form of the metric only up to the corresponding order in $\bar{z}$.
Using the definition $z=2\frac{1-\rho}{1+\rho}$ (i.e. the one that brings the pure AdS metric in FG form), the full metric $g$ up to and including $\mathcal{O}(z)$ reads

\begin{eqnarray}
g=&\frac{1}{z^2}\biggl[ 
\left(1+f_{\rho\rho}(t,\theta,\phi)z^3+\mathcal{O}(z^4)\right)dz^2 \nonumber \\
&+\mathcal{O}(z^4) dz dt+\mathcal{O}(z^4) dz d\theta+\mathcal{O}(z^4) dz d\phi \nonumber \\
&- \left(1+\frac{z^2}{2}+f_{tt}(t,\theta,\phi)z^3 +\mathcal{O}(z^4)\right)dt^2 \nonumber \\
&+ \left(2 f_{t\theta}(t,\theta,\phi)z^3+\mathcal{O}(z^4)\right)dt d\theta + \left(2 f_{t\phi}(t,\theta,\phi) z^3+\mathcal{O}(z^4)\right)dt d\phi  \nonumber \\
&+  \left(1-\frac{z^2}{2}+f_{\theta\theta}(t,\theta,\phi)z^3 +\mathcal{O}(z^4)\right)d\theta^2 \nonumber \\
&+ \left(2 f_{\theta\phi}(t,\theta,\phi) z^3+\mathcal{O}(z^4)\right)d\theta d\phi \nonumber\\
&+  \sin^2\theta \left(1-\frac{z^2}{2}+\frac{f_{\phi\phi}(t,\theta,\phi)}{\sin^2\theta} z^3 +\mathcal{O}(z^4)\right)d\phi^2
\biggr].
\end{eqnarray}
Notice that this is not yet in the FG form because the $zz$-component is not $\frac{1}{z^2}$ up to the desired $\mathcal{O}(z)$. Defining $\bar{z}=z\left(1+\frac{1}{6}z^3f_{\rho\rho}(t,\theta,\phi)\right),\bar{t}=t,\bar{\theta}=\theta,\bar{\phi}=\phi$ we finally obtain the metric in FG form up to and including $\mathcal{O}(\bar{z})$
\begin{eqnarray}
g=&\frac{1}{\bar{z}^2}\biggl\{ 
\left(1+\mathcal{O}(\bar{z}^4)\right)d\bar{z}^2 \nonumber \\
&+\mathcal{O}(\bar{z}^4) d\bar{z} d\bar{t}+\mathcal{O}(\bar{z}^4) d\bar{z} d\bar{\theta}+\mathcal{O}(\bar{z}^4) d\bar{z} d\bar{\phi} \nonumber \\
&- \left[1+\frac{\bar{z}^2}{2}+\left(f_{tt}(\bar{t},\bar{\theta},\bar{\phi})-\frac{1}{3}f_{\rho\rho}(\bar{t},\bar{\theta},\bar{\phi})\right)\bar{z}^3 +\mathcal{O}(\bar{z}^4)\right]d\bar{t}^2 \nonumber \\
&+ \left(2 f_{t\theta}(\bar{t},\bar{\theta},\bar{\phi})\bar{z}^3+\mathcal{O}(\bar{z}^4)\right)d\bar{t} d\bar{\theta} + \left(2 f_{t\phi}(\bar{t},\bar{\theta},\bar{\phi}) \bar{z}^3+\mathcal{O}(\bar{z}^4)\right)d\bar{t} d\bar{\phi}  \nonumber \\
&+  \left[1-\frac{\bar{z}^2}{2}+\left(f_{\theta\theta}(\bar{t},\bar{\theta},\bar{\phi})+\frac{1}{3}f_{\rho\rho}(\bar{t},\bar{\theta},\bar{\phi})\right)\bar{z}^3 +\mathcal{O}(\bar{z}^4)\right]d\bar{\theta}^2 \nonumber \\
&+ \left(2 f_{\theta\phi}(\bar{t},\bar{\theta},\bar{\phi}) \bar{z}^3+\mathcal{O}(\bar{z}^4)\right)d\bar{\theta} d\bar{\phi} \nonumber\\
&+   \sin^2\bar{\theta} \left[1-\frac{\bar{z}^2}{2}+\left(\frac{f_{\phi\phi}(\bar{t},\bar{\theta},\bar{\phi})}{ \sin^2\bar{\theta}}+\frac{1}{3}f_{\rho\rho}(\bar{t},\bar{\theta},\bar{\phi})\right)\bar{z}^3 +\mathcal{O}(\bar{z}^4)\right]d\bar{\phi}^2
\biggr\}.
\end{eqnarray}
Notice that $f_{\rho\rho}$ has been reabsorbed in $g_{\bar{t}\bar{t}},g_{\bar{\theta}\bar{\theta}},g_{\bar{\phi}\bar{\phi}}$.


OLD

We will be interested in the Cauchy evolution of spacetimes whose asymptotic isometry group is the same as AdS$_4$, i.e. $SO(3,2)$. These are called asymptotically AdS spacetimes. In particular, the boundary of these spacetimes is exactly the same as AdS$_4$, thus we will be dealing with an initial-boundary value problem and boundary conditions on all the fields involved in the Generalized Harmonic formulation will have to be imposed.
In this Section we present suitable boundary conditions for asymptotically Anti-de Sitter spacetimes.

Requiring an $SO(3,2)$ symmetry near the boundary amounts to imposing that the metric solution, written in the form $g=\hat{g}+h$, approaches the pure AdS metric $\hat{g}$ at a rate that satisfies the asymptotic Killing equation (in an arbitrary set of coordinates), 
%\begin{equation}\label{eqn:asyKilleq}
%(\mathcal{L}_X g)_{\alpha\beta}=\mathcal{O}(h_{\alpha\beta})\;\;\;\; \textrm{near the AdS boundary at $\rho=1$},
%\end{equation}
for any $X$ generator of $SO(3,2)$. 
In the following, we will refer to any set of coordinates in which the metric $g$ asymptotes to $\hat{g}$ in the form \eref{eqn:ads4_compact} as spherical coordinates (although they should be regarded as asymptotically spherical coordinates) and the indices associated with their coordinate basis will be chosen from the first part of the Greek alphabet, $\alpha,\beta,\gamma,\delta,\dots$
Similarly, we will refer to any set of coordinates in which $g$ asymptotes to $\hat{g}$ in the form \eref{eqn:ads4_final} as Cartesian coordinates (although they should be regarded as asymptotically Cartesian coordinates) and the indices associated with their coordinate basis will be chosen from the second part of the Greek alphabet, $\mu,\nu,\rho,\sigma,\dots$

In spherical coordinates, the solution to \eref{eqn:asyKilleq} gives the boundary conditions (see [cite 1201.2132] for the details)
\begin{eqnarray}
\label{eq:sphbounconh}
h_{\rho\alpha}&=f_{\rho\alpha}(t,\theta,\phi)(1-\rho)^2+\mathcal{O}((1-\rho)^3) \;\; \textrm{ if $\alpha\neq\rho$}, \\ \nonumber
h_{\alpha\beta}&=f_{\alpha\beta}(t,\theta,\phi)(1-\rho)+\mathcal{O}((1-\rho)^{2}) \;\; \textrm{ otherwise},
\end{eqnarray}
(where $m,n=t,\theta,\phi$) for arbitrary functions $f_{\alpha\beta}(t,\theta,\phi)$. Furthermore, we impose a boundary condition on the massless scalar field $\varphi$ that preserves the asymptotics \eref{eq:sphbounconh} during evolution (as verified in CITE [cite 1201.2132]):
\begin{equation}\label{eq:sphbounconphi}
\varphi=b(t,\theta,\phi)(1-\rho)^3+\mathcal{O}((1-\rho)^4)
\end{equation}
for arbitrary $b(t,\theta,\phi)$.

In Cartesian coordinates, these read
\begin{eqnarray}
\label{eq:carbouncondh}
h_{\mu\nu}&=&f_{\mu\nu}(t,x,y,z)|_{\rho(x,y,z)=1}(1-\rho(x,y,z))+\mathcal{O}((1-\rho(x,y,z))^{2}), \\
\label{eq:carbouncondphi}
\varphi&=&c(t,x,y,z)|_{\rho(x,y,z)=1}(1-\rho(x,y,z))^3+\mathcal{O}((1-\rho(x,y,z))^{4}), 
\end{eqnarray}
for arbitrary $f_{\mu\nu}$ and $c$.

Regarding the source functions in a given set of coordinates, involved in the Generalized Harmonic Formulation employed in this study (CITE), their near-boundary behaviour can be deduced from the behaviour of the full metric $g$ through the definition 
\begin{equation}\label{eq:defsoufunsph}
H^\alpha \equiv \Box x^\alpha = \frac{1}{\sqrt{-g}}\partial_\beta (\sqrt{-g}g^{\beta\gamma}x^\alpha_{\;\;,\gamma})=\frac{1}{\sqrt{-g}}\partial_\beta (\sqrt{-g}g^{\beta\alpha})
\end{equation}
in spherical coordinates, and similarly in Cartesian coordinates. 
In spherical coordinates, denoting the pure AdS values by $\hat{H}_\alpha$, \eref{eq:sphbounconh} implies,
\begin{eqnarray}\label{eq:sphbouncondsoufunc}
H_\alpha&=&\hat{H}_\alpha+f_\alpha(t,\theta,\phi)(1-\rho)^3+\mathcal{O}((1-\rho)^4) \;\; \textrm{ if $\alpha\neq\rho$} \\ \nonumber
H_\rho&=&\hat{H}_\rho+f_\rho(t,\theta,\phi)(1-\rho)^2+\mathcal{O}((1-\rho)^3)
\end{eqnarray}
for arbitrary $f_\alpha$.
In Cartesian coordinates,  denoting the pure AdS values by $\hat{H}_\mu$, \eref{eq:carbouncondh} implies,
\begin{equation}\label{eq:carbouncondsoufun}
H_\mu=\hat{H}_\mu+f_\mu(t,x,y,z)|_{\rho(x,y,z)=1}(1-\rho(x,y,z))^2+\mathcal{O}((1-\rho(x,y,z))^3)
\end{equation}
for arbitrary $f_\mu$.

\section{Numerical Scheme}\label{sec:numerical_scheme}

We solve the Einstein equations in generalized harmonic form~\cite{Pretorius:2004jg} with constraint damping, for asymptotically AdS spacetimes~\cite{Bantilan:2012vu} in Cartesian coordinates.
We discretize the resulting PDEs with second order finite differences, and integrate in $t$ using an iterative Newton-Gauss-Seidel relaxation procedure. 
We use the PAMR/AMRD libraries \cite{PAMR} for running these simulations in parallel on Linux computing clusters and for adaptive mesh refinement support.
We obtain initial data using a multigrid algorithm built into the PAMR/AMRD libraries.
The numerical grid is in $(t,x,y,z)$ with $t \in [0,t_{max}]$, $x \in [-1,1]$, $y \in [-1,1], z \in [-1,1]$.
The typical grid resolution gives $N_x=N_y=N_z=217$ points in each of the Cartesian directions, with equal grid spacings $\Delta x = \Delta y = \Delta z\equiv \Delta$.
We use a typical Courant factor of $\delta \equiv \Delta t / \Delta x = 0.15$.

We search for the position $R(\theta,\phi)$ of the apparent horizons (AH) using a flow method (MORE DETAILS ON THIS?). A $(\theta,\phi)\in [0,\pi)\times [0,2\pi)$ grid with resolution $N_\theta\times N_\phi=9\times 17$ is enough to find the AH in the simulation considered in Section \ref{sec:results}. We use the excision method to evolve black hole spacetimes, thereby removing geometric singularities from the computational domain. At the excision surface, the usual equations of motion are solved using one-sided stencils. 
Kreiss-Oliger dissipation~\cite{kreiss1973methods} is essential to damp unphysical high-frequency noise that arise at such grid boundaries, and we use a typical dissipation parameter of $\epsilon=0.35$.

\subsection{Cartesian Evolution Variables and Boundary Conditions}\label{subsec:cartevvarboucon}

The boundary conditions on asymptotically AdS spacetimes, found in Section~\ref{subsec:asyAdS}, can easily be imposed as Dirichlet boundary conditions at the boundary $\rho=1$ if we appropriately define and evolve a new set of variables, from which the full solution $(g_{\mu\nu},\varphi,H_\mu)$ can subsequently be reconstructed. %Even though the code employs Cartesian coordinates, we define these variables also in spherical coordinates for completeness and because they will turn out to be useful in Section~\ref{sec:bouset2} .
These variables will be here obtained for Cartesian coordinates only, as these are the coordinates employed in the code that we wish to present. However, in Section~\ref{sec:bouset2} we choose to show expressions for quantities at the AdS boundary in spherical coordinates, therefore in~\ref{sec:sphevvarboucon} we construct the spherical version of the evolution variables and make the relation to their Cartesian analog explicit.

The Cartesian metric evolution variables, $\bar{g}_{\mu\nu}$, are defined by (i) considering the deviation from pure AdS in Cartesian coordinates $h_{\mu\nu}=g_{\mu\nu}-\hat{g}_{\mu\nu}$, (ii) stripping $h_{\mu\nu}$ of as many factors of $(1-\rho^2)$ as needed so that each components falls off linearly in $(1-\rho)$ near the AdS boundary at $\rho=1$\footnote{Looking at the boundary conditions \eref{eq:carbouncondh}, it seems natural to factor out $(1-\rho)$ rather than $(1-\rho^2)$. However, the latter is preferred since it preserves the even/odd character of the components of $h$.}. 
We see from \eref{eq:carbouncondh} that 
\begin{equation}\label{eq:gbarcart}
\bar{g}_{\mu\nu}=h_{\mu\nu}.
\end{equation}

Similarly, the Cartesian boundary condition on the scalar field \eref{eq:carbouncondphi} suggests that we use the evolution variable
\begin{equation}
\bar{\varphi}=\frac{\varphi }{(1-\rho^2)^2}.
\end{equation}

Finally, the boundary conditions \eref{eq:carbouncondsoufun} on $H_\mu$ suggest the use of
\begin{equation}\label{eq:soufunb}
\bar{H}_\mu=\frac{H_\mu-\hat{H}_\mu}{1-\rho^2 }.
\end{equation}

%The evolved variables $\bar{g}_{\mu\nu}$ are constructed out of the full metric $g_{\mu\nu}$ and the pure AdS metric $\hat{g}_{\mu\nu}$ by $g_{\mu\nu} = \hat{g}_{\mu\nu} + \bar{g}_{\mu\nu}$. 
%The evolved variables $\bar{g}_{\mu\nu}$ are constructed out of the full metric $g_{\mu\nu}$ by considering the deviations from the pure AdS metric $h_{\mu\nu}=g_{\mu\nu} - \hat{g}_{\mu\nu}$ and stripping these of as many factors of $(1-\rho^2)$ as necessary so that they fall off linearly in $(1-\rho)$ near the AdS boundary $\rho=1$. %their leading order in the near-boundary expansion is $\mathcal{O}(q)$. 
%From the boundary conditions \eref{eq:carbouncondh}, we immediately see that $h_{\mu\nu}=\bar{g}_{\mu\nu}$ in Cartesian coordinates, so $g_{\mu\nu} = \hat{g}_{\mu\nu}+\bar{g}_{\mu\nu}$.
%Similarly, the evolved scalar field variable $\bar{\varphi}$ is constructed out of a real scalar field $\varphi$ by
%$\varphi = (1-\rho^2)^2 \bar{\varphi}$.
%Following the same argument, the evolved source function variables $\bar{H}_\mu$ are constructed out of the full generalized harmonic source functions $H_\mu$ and the values they take on in pure AdS by $\hat{H}_\mu$ by $H_\mu = \hat{H}_\mu + (1-\rho^2) \bar{H}_\mu$.

%By construction, the evolved variables $\bar{g}_{\mu \nu}$, $\bar{\varphi}$, $\bar{H}_{\mu}$ all fall off linearly in $(1-\rho)$ near the AdS boundary $\rho=1$. 
For evolved variables defined in this way, the boundary conditions \eref{eq:carbouncondh},\eref{eq:carbouncondphi}, \eref{eq:carbouncondsoufun} can be easily imposed as reflective Dirichlet boundary conditions at the AdS boundary $\rho=1$ as $\bar{g}_{\mu\nu}|_{\rho=1}=0,\bar{\varphi}|_{\rho=1}=0,\bar{H}_\mu|_{\rho=1}=0$.
%We set reflective Dirichlet boundary conditions at the AdS boundary $\rho=1$ for all evolved variables. 
The only complication comes from the fact that in Cartesian coordinates $\rho^2=x^2+y^2+z^2$, so the AdS boundary generally does not lie on a Cartesian grid point. 
For any given evolution variable, we thus take its Dirichlet boundary condition at $\rho=1$ and its values at points further to the interior $\rho<1$, and use these to set its value at grid points that are at most one grid point away from $\rho=1$ by linear interpolation. 



\section{Gauge Choice}\label{sec:gauge_choice}

The statement that we use Cartesian coordinates completely defines the coordinates only at the AdS boundary, as emphasized in Section~\ref{subsec:asyAdS}. Coordinates over the entire spacetime are fully determined only once we specify a gauge choice of source functions $H_\mu$. As we will see, this choice cannot be completely arbitrary near the boundary if we wish to achieve stable evolution. In this Section we sketch the procedure for finding a possible generalized harmonic gauge choice that has proven to provide stable Cauchy 3+1 evolution.
%In this Section we sketch the procedure for finding the generalized harmonic gauge choice that achieves stable Cauchy 3+1 evolution for asymptotically AdS$_4$ spacetimes.
For a more detailed discussion in a simpler context with more symmetry, see~\cite{Bantilan:2012vu}. 

The first step involves expanding the evolved variables $\bar{g}_{\mu \nu}$, $\bar{H}_{\mu}$, $\bar{\varphi}$ in power series about $(1-\rho(x,y,z)) \equiv q(x,y,z) = 0$. By definition, they are linear in $q$:
\begin{eqnarray}\label{eqn:qexp}
\bar{g}_{\mu \nu} &=& \bar{g}_{(1) \mu \nu} q + \bar{g}_{(2) \mu \nu} q^2 + \bar{g}_{(3) \mu \nu} q^3 + \mathcal{O}(q^4) \nonumber \\
\bar{H}_{\mu} &=& \bar{H}_{(1) \mu} q + \bar{H}_{(2) \mu} q^2 + \bar{H}_{(3) \mu} q^3 + \mathcal{O}(q^4) \nonumber \\
\bar{\varphi} &=& \bar{\varphi}_{(1)} q + \bar{\varphi}_{(2)} q^2 + \bar{\varphi}_{(3)} q^3 + \mathcal{O}(q^4),
\end{eqnarray}
where all the coefficients are functions of $(t,x,y,z)$, restricted on the surface $\rho(x,y,z)\equiv\sqrt{x^2+y^2+z^2}=1$ (or $q(x,y,z)=0$). We now plug this into the evolution equations \eref{eqn:efe_gh_modified} and we expand each component in powers of $q$. The three lowest orders, $q^{-2},q^{-1},q^0$, come from the pure AdS metric $\hat{g}$, which satisfies \eref{eqn:efe_gh_modified}, so they vanish trivially (while the remaining orders vanish only if $\bar{g}_{\mu \nu}$, $\bar{H}_{\mu}$, $\bar{\varphi}$ are a solution of \eref{eqn:efe_gh_modified}).

We are now interested in identifying the order of $q$ at which the second derivatives of $ \bar{g}_{(1) \mu \nu}$ w.r.t. $(t,x,y,z)$ evaluated on $\rho(x,y,z)=1$ appear.
For each component, we denote their combination by $\tilde{\Box}\bar{g}_{(1)\mu\nu}$, i.e. 
\begin{equation}
\tilde{\Box}\bar{g}_{(1)\mu\nu}\equiv\left.\biggl(c^t_{\mu\nu}\frac{\partial^2}{\partial t^2}+c^x_{\mu\nu}\frac{\partial^2}{\partial x^2}+c^y_{\mu\nu}\frac{\partial^2}{\partial y^2}+c^z_{\mu\nu}\frac{\partial^2}{\partial z^2}\biggr)\right |_{\rho(x,y,z)=1}\bar{g}_{(1)\mu\nu},
\end{equation}
for some functions $c^t_{\mu\nu},c^x_{\mu\nu},c^y_{\mu\nu},c^z_{\mu\nu}$ of $(t,x,y,z)$ at $\rho(x,y,z)=1$ (clearly none of these coefficients are tensors, despite the notation, and there is no sum over repeated indices).
These derivative terms are included in the first piece of \eref{eqn:efe_gh_modified}, $-\frac{1}{2}g^{\rho \sigma} g_{\mu \nu, \rho \sigma}$, more precisely in $-\frac{1}{2}g^{\rho \sigma} \bar{g}_{\mu \nu, \rho \sigma}$.
Therefore we can easily find their order by recalling that the leading order of the inverse metric is given by its purely AdS piece, $g^{\mu\nu}=\mathcal{O}(\hat{g}^{\mu\nu})=\mathcal{O}(q^{2})$, and that $\bar{g}_{(1)\mu\nu}$ is multiplied by $q$ in the near-boundary expression of $\bar{g}_{\mu\nu}$ (see the first of \eref{eqn:qexp}). Therefore $\tilde{\Box}\bar{g}_{(1)\mu\nu}$ must appear in the coefficient of order $q^{3}$ for every component of \eref{eqn:efe_gh_modified}.

Using all this, each component of the expansion near $q=0$ of the Einstein equations \eref{eqn:efe_gh_modified} can be written in the schematic form
\begin{equation}\label{eq:efefullexp}
A_{(1)\mu\nu}q+A_{(2)\mu\nu}q^2+(\tilde{\Box}\bar{g}_{(1)\mu\nu}+B_{(3)\mu\nu})q^3+A_{(4)\mu\nu}q^4+\mathcal{O}(q^5)=0
\end{equation}
or, rearranging the terms in order to obtain wave-like equations,
\begin{equation}
\tilde{\Box}\bar{g}_{(1)\mu\nu}=-A_{(1)\mu\nu}\frac{1}{q^2}-A_{(2)\mu\nu}\frac{1}{q}-B_{(3)\mu\nu}-A_{(4)\mu\nu}q+\mathcal{O}(q^2).
\end{equation}

Here we write these components explicitly including only the leading order term. Operatively, the near-boundary expansion can be obtained easily if we first write the Cartesian coordinates $(x,y,z)$ in terms of the boundary-adapted spherical coordinates $(q,\theta,\phi)$ before expanding near $q=0$. We get
\begin{eqnarray}\label{eqn:efett}
\tilde{\Box}\bar{g}_{(1)tt}=&-(\cos \theta (3 \cos \theta  \bar{g}_{(1)xx}-2 \bar{H}_{(1)x}) \nonumber \\
&+\sin\theta (3 \sin \theta \cos^2\phi \bar{g}_{(1) yy}+3
   \sin \theta  \sin \phi (2 \cos \phi  \bar{g}_{(1) yz}+\sin\phi
   \bar{g}_{(1) zz}) \nonumber \\
   &-2 (\cos \phi \bar{H}_{(1) y}+\sin\phi
   \bar{H}_{(1) z}))+3 \sin 2 \theta  \cos \phi \bar{g}_{(1) xy}+3
   \sin 2 \theta  \sin \phi \bar{g}_{(1) xz})q^{-2} \nonumber \\
&+\mathcal{O}(q^{-1}).
\end{eqnarray}
\begin{eqnarray}\label{eqn:efetx}
\tilde{\Box}\bar{g}_{(1)tx}=&-2 \cos \theta (3 \cos\theta \bar{g}_{(1) tx}+3 \sin \theta
   (\cos \phi  \bar{g}_{(1) ty}+\sin \phi  \bar{g}_{(1)tz})-2
   \bar{H}_{(1) t})    q^{-2} \nonumber \\
&+\mathcal{O}(q^{-1}).
\end{eqnarray}
\begin{eqnarray}\label{eqn:efety}
\tilde{\Box}\bar{g}_{(1)ty}=&-2 \cos \phi \sin\theta (3 \cos\theta \bar{g}_{(1) tx}+3 \sin \theta
   (\cos \phi  \bar{g}_{(1) ty}+\sin \phi  \bar{g}_{(1)tz})-2
   \bar{H}_{(1) t})    q^{-2} \nonumber \\
&+\mathcal{O}(q^{-1}).
\end{eqnarray}
\begin{eqnarray}\label{eqn:efetz}
\tilde{\Box}\bar{g}_{(1)tz}=&-2 \sin \theta \sin\phi (3 \cos\theta \bar{g}_{(1) tx}+3 \sin \theta
   (\cos \phi \bar{g}_{(1) ty}+\sin \phi  \bar{g}_{(1)tz})-2
   \bar{H}_{(1) t})    q^{-2} \nonumber \\
&+\mathcal{O}(q^{-1}).
\end{eqnarray}
\begin{eqnarray}\label{eqn:efexx}
\tilde{\Box}\bar{g}_{(1)xx}=&\frac{1}{4} (3 (-4 \cos ^2\theta (\bar{g}_{(1) tt}+2 \bar{g}_{(1)
   xx})+(\cos 2 \theta +3) (\bar{g}_{(1) yy}+\bar{g}_{(1)
zz}) \nonumber \\
&+8 \cos \theta  \bar{H}_{(1) x})-8 \sin \theta  \cos \phi 
   (3 \cos \theta  \bar{g}_{(1)xy}+\bar{H}_{(1) y}) \nonumber \\
   &-8 \sin\theta \sin\phi (3 \cos\theta \bar{g}_{(1) xz}+\bar{H}_{(1) z}) \nonumber \\
   &+6 \sin^2\theta  \cos 2 \phi  (\bar{g}_{(1)yy}-\bar{g}_{(1)zz})+12
   \sin^2\theta  \sin 2 \phi  \bar{g}_{(1) yz})    q^{-2} \nonumber \\
&+\mathcal{O}(q^{-1}).
\end{eqnarray}
\begin{eqnarray}\label{eqn:efexy}
\tilde{\Box}\bar{g}_{(1)xy}=&-\frac{1}{2} (2 \sin \theta  \cos \phi  (3 \cos \theta (\bar{g}_{(1)tt}+\bar{g}_{(1)xx}+\bar{g}_{(1)yy}-\bar{g}_{(1)zz})-4 \bar{H}_{(1) x}) \nonumber \\
&+3 \bar{g}_{(1)xy} (2 \cos 2\theta  \sin ^2\phi +\cos 2 \phi +3)+6 \sin ^2\theta  \sin 2 \phi 
   \bar{g}_{(1)xz} \nonumber \\
   &+6 \sin 2 \theta  \sin \phi  \bar{g}_{(1)yz}-8 \cos
   \theta  \bar{H}_{(1) y})    q^{-2} \nonumber \\
&+\mathcal{O}(q^{-1}).
\end{eqnarray}
\begin{eqnarray}\label{eqn:efexz}
\tilde{\Box}\bar{g}_{(1)xz}=&- (\sin \theta  \sin \phi  (3 \cos \theta  (\bar{g}_{(1)tt}+\bar{g}_{(1)xx}-\bar{g}_{(1)yy}+\bar{g}_{(1)zz})-4
   \bar{H}_{(1) x}) \nonumber \\
   &+3 \sin ^2\theta \sin 2 \phi  \bar{g}_{(1)xy}-3 \sin
   ^2\theta  \cos 2 \phi  \bar{g}_{(1)xz} \nonumber \\
   &+\frac{3}{2} (\cos 2 \theta +3)
   \bar{g}_{(1)xz}+3 \sin 2 \theta  \cos \phi  \bar{g}_{(1)yz}-4 \cos
   \theta  \bar{H}_{(1)z})    q^{-2} \nonumber \\
&+\mathcal{O}(q^{-1}).
\end{eqnarray}
\begin{eqnarray}\label{eqn:efeyy}
\tilde{\Box}\bar{g}_{(1)yy}=&-( (\sin \theta (3 \sin \theta  (2 \cos ^2\phi  \bar{g}_{(1)yy}+\sin 2 \phi  \bar{g}_{(1)yz}-\bar{g}_{(1)zz}) \nonumber \\
&-6 \cos\phi \bar{H}_{(1) y}+2 \sin \phi  \bar{H}_{(1) z}) \nonumber \\
&+6 \sin \theta \cos
   \theta \cos \phi  \bar{g}_{(1)xy}-6 \sin \theta  \cos \theta  \sin   \phi  \bar{g}_{(1)xz}+2 \cos \theta  \bar{H}_{(1) x}) \nonumber \\
   &-3 \sin ^2\theta \cos ^2\phi  \bar{g}_{(1)tt}+\frac{3}{4} \bar{g}_{(1)xx} (2 \sin ^2\theta  \cos 2 \phi +\cos 2 \theta +3)  )  q^{-2} \nonumber \\
&+\mathcal{O}(q^{-1}).
\end{eqnarray}
\begin{eqnarray}\label{eqn:efeyz}
\tilde{\Box}\bar{g}_{(1)yz}=&-\frac{1}{2} \sin \theta (4 \sin \phi  (3 \cos \theta  \bar{g}_{(1)xy}-2 \bar{H}_{(1) y})+4 \cos \phi  (3 \cos \theta  \bar{g}_{(1)xz}-2 \bar{H}_{(1) z}) \nonumber \\
&+3 \sin \theta  \sin 2 \phi  (\bar{g}_{(1)tt}-\bar{g}_{(1)xx}+\bar{g}_{(1)yy}+\bar{g}_{(1)zz})+12 \sin \theta  \bar{g}_{(1)yz})  q^{-2} \nonumber \\
&+\mathcal{O}(q^{-1}).
\end{eqnarray}
\begin{eqnarray}\label{eqn:efezz}
\tilde{\Box}\bar{g}_{(1)zz}=&(-2 \cos \theta  (3 \sin \theta  \sin \phi  \bar{g}_{(1)xz}+\bar{H}_{(1)x}) \nonumber \\
&+ \sin \theta  (3 \sin \theta  \bar{g}_{(1)yy}-6 \sin\theta  \sin \phi  (\cos \phi  \bar{g}_{(1)yz}+\sin \phi 
   \bar{g}_{(1)zz}) \nonumber \\
   &-2 \cos \phi  \bar{H}_{(1) y} +6 \sin \phi  \bar{H}_{(1)z}) -3  \sin ^2\theta  \sin ^2\phi  \bar{g}_{(1)tt} \nonumber \\
   &+\frac{3}{4}  \bar{g}_{(1)xx} (-2 \sin ^2\theta  \cos 2 \phi +\cos 2 \theta
   +3) +3 \sin 2 \theta  \cos \phi  \bar{g}_{(1)xy})  q^{-2} \nonumber \\
&+\mathcal{O}(q^{-1}).
\end{eqnarray}
All that remains is to write down the generalized harmonic constraints $0=C_\mu \equiv H_\mu-\Box x_\mu$ in the same near-boundary expansion
\begin{eqnarray}\label{eqn:ct}
C_t=&q^2 (-3 \cos \theta  \bar{g}_{(1)tx}-3 \sin \theta  \cos \phi  \bar{g}_{(1)ty}-3 \sin \theta  \sin \phi \bar{g}_{(1)tz}+2
   \bar{H}_{(1) t}) \nonumber \\
   &+\mathcal{O}(q^3),
\end{eqnarray}
\begin{eqnarray}\label{eqn:cx}
C_x=&\frac{1}{2} q^2 (-3 \cos \theta  \bar{g}_{(1)tt}-3 \cos \theta  \bar{g}_{(1)xx}-6 \sin \theta  \cos \phi  \bar{g}_{(1)xy}-6 \sin
   \theta  \sin \phi  \bar{g}_{(1)xz} \nonumber \\
   &+3 \cos \theta  \bar{g}_{(1)yy}+3
   \cos \theta  \bar{g}_{(1)zz}+4 \bar{H}_{(1) x}) \nonumber \\
   &+\mathcal{O}(q^3),
\end{eqnarray}
\begin{eqnarray}\label{eqn:cy}
C_y=&\frac{1}{2} q^2 (-3 \sin \theta  \cos \phi  \bar{g}_{(1)tt}+3 \sin
   \theta  \cos \phi  \bar{g}_{(1)xx}-6 \cos \theta  \bar{g}_{(1)xy} \nonumber \\
   &-3
   \sin \theta  \cos \phi  \bar{g}_{(1) yy}-6 \sin \theta  \sin \phi
   \bar{g}_{(1)yz}+3 \sin \theta  \cos \phi  \bar{g}_{(1)zz}+4
   \bar{H}_{(1) y}) \nonumber \\
   &+\mathcal{O}(q^3),
\end{eqnarray}
\begin{eqnarray}\label{eqn:cz}
C_z=&\frac{1}{2} q^2 (-3 \sin \theta \sin \phi  \bar{g}_{(1)tt}+3 \sin
   \theta \sin \phi \bar{g}_{(1)xx}-6 \cos \theta  \bar{g}_{(1)xz} \nonumber \\
   &+3
   \sin \theta \sin \phi  \bar{g}_{(1)yy}-6 \sin \theta  \cos \phi    \bar{g}_{(1)yz}-3 \sin \theta  \sin \phi  \bar{g}_{(1)zz}+4
   \bar{H}_{(1)z}) \nonumber \\
   &+\mathcal{O}(q^3),
\end{eqnarray}
where the coordinates $(q,\theta,\phi)$ should be intended as functions of $(x,y,z)$.

In the generalized harmonic formulation, choosing a gauge amounts to choosing a set of generalized harmonic source functions $H_\mu$ for the entire evolution.
The criterion for our choice is the following: we pick source functions to explicitly eliminate all $q^{-2}$ terms that appear in \eref{eqn:efett}-\eref{eqn:efezz}, provided that the generalized harmonic constraints \eref{eqn:ct}-\eref{eqn:cz} are satisfied.
A simple (but clearly not unique) gauge choice satisfying this criterion has leading order coefficients given by

%As long as the generalized harmonic constraints \eref{eqn:ct}-\eref{eqn:cz} are satisfied, a gauge choice that explicitly eliminates all $q^{-2}$ terms that appear in \eref{eqn:efett}-\eref{eqn:efezz} has leading order coefficients that satisfy
\begin{eqnarray}\label{eqn:target_gauge_xyz}
\bar{H}_{(1)x}&=&\frac{3}{2\sqrt{x^2+y^2+z^2}}(x \bar{g}_{(1)xx}+y\bar{g}_{(1)xy}+z\bar{g}_{(1)xz}) \nonumber \\
\bar{H}_{(1)y}&=&\frac{3}{2\sqrt{x^2+y^2+z^2}}(x \bar{g}_{(1)xy}+y\bar{g}_{(1)yy}+z\bar{g}_{(1)yz}) \nonumber \\
\bar{H}_{(1)z}&=&\frac{3}{2\sqrt{x^2+y^2+z^2}}(x \bar{g}_{(1)xz}+y\bar{g}_{(1)yz}+z\bar{g}_{(1)zz}).
\end{eqnarray}
This can be immediately seen from the fact that the $q^{-2}$ order of $\tilde{\Box}\bar{g}_{(1)t\mu}$ for $\mu\neq t$ is proportional to the leading order of $C_t$ (for any choice of $\bar{H}_\mu$) while, for our particular choice of $\bar{H}_\mu$, the $q^{-2}$ order of the remaining $\tilde{\Box}\bar{g}_{(1)\mu\nu}$ and the leading orders of $C_x,C_y,C_z$ depend only on the factor $\bar{g}_{(1)tt}-\bar{g}_{(1)xx}-\bar{g}_{(1)yy}-\bar{g}_{(1)zz}$.

Finally, for the t-component we choose
\begin{equation}\label{eqn:target_gauge_t}
\bar{H}_{(1)t}=\frac{3}{2\sqrt{x^2+y^2+z^2}}(x \bar{g}_{(1)tx}+y\bar{g}_{(1)ty}+z\bar{g}_{(1)tz}),
\end{equation}
which makes the leading order of \eref{eqn:ct} vanish trivially.
This is the asymptotic gauge condition that we have empirically verified to lead to stable 3+1 evolution. Notice that this choice does not fix the values of $H_\mu$ in the bulk, which are still arbitrary. Our choice is made explicit in \ref{sec:GCbulk}.

It is important to develop an understanding about the reason why the choice of $\bar{H}_\mu$ is not completely free (despite the fact that this is simply a gauge choice in the generalized harmonic formulation) and the criterion above is expected to work. 
Let us notice that \eref{eq:efefullexp} implies that $A_{(i)\mu\nu}=0$ for all $i\neq3$ for a solution of the full evolution equations, for any choice of source functions.
%
%So, we expect order $q^{-2}$ and $q^{-1}$ of the right hand sides of \eref{eqn:efett}-\eref{eqn:efezz} to vanish for the solution of the full evolution equations, for any choice of source functions. 
However, for some values of $\bar{H}_\mu$ a solution consistent with the boundary conditions might not exist. To see this explicitly in a simple way, it is more convenient to switch to radial coordinates $(t,\rho,\theta,\phi)$ and look at the near-boundary expansion of the $tt$-component of the Einstein equations:
\begin{eqnarray}\label{eqn:efettsph}
\tilde{\Box}\bar{g}_{(1)tt}=&(-3 \bar{g}_{(1)\rho\rho}+2\bar{H}_{(1)\rho})q^{-2}+\mathcal{O}(q^{-1}).
\end{eqnarray}
Suppose now to choose a gauge that becomes harmonic with respect to pure AdS after some time $t_0$ in the evolution, i.e. $\bar{H}_{\alpha}(t>t_0)=0$ for some $t_0$. Since we argued above that the $q^{-2}$ term must vanish, \eref{eqn:efettsph} immediately tells us that we must have $\bar{g}_{(1)\rho\rho}(t>t_0)=0$, i.e. $\bar{g}_{(1)\rho\rho}(t>t_0)=0$ must go to 0 as we approach the boundary faster than what the boundary conditions \eref{eq:sphbounconh} impose. This does not restrict the evolution in any way, as shown in \ref{subsec:asyAdS} by the fact that $\bar{g}_{(1)\rho\rho}(t>t_0)=0$ can be absorbed in a re-definition of the coordinates near the boundary. However, in practice the code would fail to find such a solution since the Dirichlet boundary condition $\bar{g}_{(1)\rho\rho}|_{\rho=1}=0$ would not enforce $\bar{g}_{(1)\rho\rho}=0$ when needed, letting that metric component differ from zero by the solution error.

We see that the imposition of boundary conditions, required by the initial-boundary value problem, is the reason why the gauge choice in asymptotically AdS spacetimes is restricted. We also see that our criterion "helps" evolution to cancel orders $q^{-2}$ of $\tilde{\Box}\bar{g}_{(1)\mu\nu}$, so it is clearly not in contrast with the requirements imposed by the boundary conditions and it allows a solution to be found. This also shows that our choice of source functions is far from being unique: we expect to be able to find a solution for any gauge choice that is not in contrast with the boundary conditions, regardless of whether this choice makes the orders $q^{-2}$ of $\tilde{\Box}\bar{g}_{(1)\mu\nu}$ vanish or not.

This discussion exposes another interesting point regarding the expected order of convergence near the boundary. As mentioned in the previous paragraph, the solution will, in particular, make the orders $q^{-2}$ of $\tilde{\Box}\bar{g}_{(1)\mu\nu}$ vanish. However, this is a statement about the analytic solution, which holds for the numerical one only up to solution error. Denoting by $h$ the grid spacing near the boundary for a certain resolution, the solution error is proportional to $h^n$ for a $n^{th}-$order finite difference scheme. Since the closest point to the boundary has $q=h$, the orders $q^{-2}$ of $\tilde{\Box}\bar{g}_{(1)\mu\nu}$ will be proportional to $h^{n-2}$ for the numerical solution of a $n^{th}-$order finite difference scheme. Therefore we would expect convergence at order $n-2$ near the boundary. This is not what we observe: for a second order difference scheme, we obtain second order convergence over the entire grid (as shown in~\ref{sec:convbulk}). Possible reasons for this were suggested in CITE 1201.2132v3: the misleading picture arising from the near-boundary expansion of the evolution equations as opposed to the correct description coming from the complete evolution, or hidden benefits of our gauge choice. However, this point is still lacking a thorough explanation.

\section{Boundary scalar field and stress-energy tensor}
\label{sec:bouset2}

In the code we output the scalar field and the components of the stress-energy tensor of the conformal field theory (CFT) at the time-like AdS boundary. In this Section, we obtain the analytic expression of these quantities in spherical coordinates $x^\alpha=(t,\rho,\theta,\phi)$, as they are adapted to the AdS boundary. Thus, in order to obtain their numerical values, we will have to convert the evolution variables in Cartesian coordinates $\bar{g}_{\mu\nu}$ provided by the code into their spherical coordinate version $\bar{g}_{\alpha\beta}$, defined in~\ref{sec:sphevvarboucon}, through the transformation \eref{eq:cartosph}.

To compute the scalar field of the boundary CFT, $\langle \bar{\varphi}\rangle_{CFT}$, consider first the restriction of the bulk scalar field $\bar{\varphi}$ at a time-like hypersurface $\partial M_q$ at fixed $q=1-\rho$, which we denote by $\;^{(q)}\bar{\varphi}$. We then have 
\begin{equation}
\langle \bar{\varphi}\rangle_{CFT}=\lim_{q\to0}\frac{1}{q}  \;^{(q)}\bar{\varphi}.
\end{equation}

%Their expressions are more easily obtained and displayed in spherical coordinates $x^\alpha=(t,\rho,\theta,\phi)$, defined in the usual way in terms of the Cartesian coordinates used in the code, as they are adapted to the AdS boundary.
Similarly, to compute the stress-energy tensor of the boundary CFT, $\langle T_{ij}\rangle_{CFT}$, we first compute the quasi-local stress-energy tensor $^{(q)}T_{\alpha\beta}$ at $\partial M_q$ as prescribed in  [CITE hep-th/9902121]. We have
\begin{equation}
\label{eq:qslocset}
^{(q)}T_{\alpha\beta}=\frac{1}{8\pi}\biggl(\;   \Theta_{\alpha\beta}-\Theta \gamma_{\alpha\beta}-\frac{2}{L}\gamma_{\alpha\beta}+L \;G_{\alpha\beta} \biggr),
\end{equation}
where $\Theta_{\alpha\beta}=-\gamma^\alpha_{\mu}\gamma^\beta_\nu\nabla_{(\alpha}S_{\beta)}$ is the extrinsic curvature of $\partial M_q$, $\gamma_{\mu\nu}=g_{\mu\nu}-S_\mu S_\nu$ is the induced metric on $\partial M_q$, $S^\mu$ is the spacelike, outward pointing unit vector normal to $\partial M_q$ and $G_{\mu\nu}$ is the Einstein tensor defined on $\partial M_q$. (Notice the different sign in the last term of \eref{eq:qslocset} w.r.t. to  [CITE hep-th/9902121] ). We will be interested in the value of $^{(q)}T_{\mu\nu}$ for $q$ close to 0, i.e. near the boundary.
Restricting to the 3 coordinates on the AdS boundary at $\rho=1$, i.e. $x^i=(t,\theta,\phi)$, this enables us to compute the boundary stress-energy tensor as
\begin{equation}
\langle T_{ij}\rangle_{CFT}=\lim_{q\to0}\frac{1}{q}  \;^{(q)}T_{ij}.
\end{equation}
Indices $i,j,k,\dots$ on the AdS boundary are lowered or raised by the metric of the AdS boundary, $\gamma=-dt^2+d\theta^2+\sin^2\theta d\phi^2$.

From $^{(q)}T_{\mu\nu}$ we also compute the AdS mass as follows [CITE hep-th/9902121]. At each time $t$ of evolution, we take a spacelike hypersurface $\Sigma$ in $\partial M_q$, with induced 3-metric $\sigma_{\mu\nu}=\Sigma_{\mu\nu}-u_\mu u_\nu$, where $u^\mu$ is the future pointing unit vector normal to $\Sigma$ in $\partial M_q$, lapse $N$ and shift $N^i$. The AdS mass is
\begin{equation}
M_{AdS}=\lim_{q\to0}\int_\Sigma d\theta d\phi \sqrt{\sigma} N ( ^{(q)}T_{\mu\nu} u^\mu u^\nu).
\end{equation}

In terms of the leading order coefficient of $\bar{g}_{\alpha\beta}$, setting $L=1$, for the stress-energy tensor we obtain
%variables $\bar{g}_{\mu\nu}$ in Section~\ref{sec:numerical_scheme}: we consider the deviation from pure AdS in spherical coordinates, $h_{\alpha\beta}=g_{\alpha\beta}-\bar{g}_{\alpha\beta}$, stripped of as many factors of $(1-\rho^2)$ as necessary to make their fall off near the boundary linear in $q$. Notice that $\bar{g}_{\alpha\beta}$ and $\bar{g}_{\mu\nu}$ are not in general components of the same tensor, therefore the usual transformation between tensor components in different sets of coordinates cannot be applied. The correct transformation can be easily deduced from \eref{eq:sphbouncondh} and \eref{eq:carbouncondh}: $\bar{g}_{\rho\alpha}=\frac{1}{(1-\rho^2)}\frac{\partial x^\mu}{\partial \rho}\frac{\partial x^\nu}{\partial x^\alpha}\bar{g}_{\mu\nu}$ if $\alpha\neq\rho$, $\bar{g}_{\alpha\beta}=\frac{\partial x^\mu}{\partial x^\alpha}\frac{\partial x^\nu}{\partial x^\beta}\bar{g}_{\mu\nu}$ otherwise.
\begin{eqnarray}
\langle T_{tt}\rangle_{CFT}&=&\frac{1}{16\pi} \biggl(2\bar{g}_{(1)\rho\rho}+3\bar{g}_{(1)\theta\theta}+3\frac{\bar{g}_{(1)\phi\phi}}{\sin^2\theta}\biggr) \nonumber \\
\langle T_{t\theta}\rangle_{CFT}&=&\frac{3}{16\pi}\bar{g}_{(1)t\theta} \nonumber \\
\langle T_{t\phi}\rangle_{CFT}&=&\frac{3}{16\pi}\bar{g}_{(1)t\phi} \nonumber \\
\langle T_{\theta\theta}\rangle_{CFT}&=&\frac{1}{16\pi} \biggl(3\bar{g}_{(1)tt}-2\bar{g}_{(1)\rho\rho}-3\frac{\bar{g}_{(1)\phi\phi}}{\sin^2\theta}\biggr) \nonumber \\
\langle T_{\theta\phi}\rangle_{CFT}&=&\frac{3}{16\pi}\bar{g}_{(1)\theta\phi} \nonumber \\
\langle T_{\phi\phi}\rangle_{CFT}&=&\frac{\sin^2\theta}{16\pi} (3\bar{g}_{(1)tt}-2\bar{g}_{(1)\rho\rho}-3\bar{g}_{(1)\theta\theta}), \nonumber \\
\end{eqnarray}
while for the AdS mass
\begin{equation}
M_{AdS}=\int_0^\pi d\theta \int_0^{2\pi}d\phi\frac{\sin\theta}{16\pi} \biggl(2\bar{g}_{(1)\rho\rho}+3\bar{g}_{(1)\theta\theta}+3\frac{\bar{g}_{(1)\phi\phi}}{\sin^2\theta}\biggr) \nonumber. \\
\end{equation}
From the stress-energy tensor, we can solve the eigenvalue problem $\langle T^i_{\;\;j}\rangle_{CFT} v^j=\lambda_v v^i$ at each point of the AdS boundary in order to obtain the energy density of the boundary CFT, $\epsilon_{CFT}$, as the eigenvalue associated with the unique (up to rescaling) time-like vector, and the anisotropy $(\Delta p)_{CFT}\equiv|p_1-p_2|$, where $p_1$ and $p_2$ are the eigenvalues associated, respectively, with the remaining two spacelike eigenvectors.

One useful quantity to compute is the trace of the stress-energy tensor, $\langle trT\rangle_{CFT}=\gamma^{ij} \langle T_{ij}\rangle_{CFT}$. We obtain
\begin{equation}
\langle trT\rangle_{CFT}=\frac{3}{8\pi}\biggl(\bar{g}_{(1)tt}-\bar{g}_{(1)\rho\rho}-\bar{g}_{(1)\theta\theta}-\frac{\bar{g}_{(1)\phi\phi}}{\sin^2\theta}\biggr)
\end{equation}
If we convert the spherical quantities into Cartesian ones (and we evaluate the expression at $\rho=1$), we see that $\langle trT\rangle_{CFT}$ depends only on the factor $\bar{g}_{(1)tt}-\bar{g}_{(1)xx}-\bar{g}_{(1)yy}-\bar{g}_{(1)zz}$, as for $C_x,C_y,C_z$. This is an important sanity check: the validity of the generalized harmonic constraints analitycally ensures the vanishing of the trace of the boundary stress-energy tensor, as expected for the boundary CFT.
Another important test that we performed is conservation of the analytic form of $\langle T_{ij}\rangle_{CFT}$. The proof of this is conceptually straightforward but lenghty, so it is not presented here. We will instead show in Section~\ref{sec:resbouset} that conservation is preserved numerically during evolution. Since this test is computationally expensive, we will do it only for 4 representative time steps.

Since the AdS boundary generally does not lie on points of the Cartesian grid, to obtain the values of quantities at the boundary we use first order extrapolation from bulk point values (see~\ref{sec:extrapconvbdy} for the details on how this is implemented).

\section{Results}\label{sec:results}

In this Section we discuss the most evident features of the long-time evolution of time-symmetric initial data in the presence of a scalar field with Gaussian profile, distorted along each Cartesian direction and centred at $x=y=z=0$:
\begin{eqnarray}
\label{eq:scaGaupro}
\bar{\varphi}&=&A e^{-(\tilde{r}(x,y,z)/\Delta)^2},\\
\tilde{r}(x,y,z)&\equiv&\sqrt{ x^2(1-e_x^2)+ y^2(1-e_y^2)+ z^2(1-e_z^2)}, \nonumber
\end{eqnarray}
where the amplitude of the profile is $A=0.55$, the eccentricities are $e_x=0.3, e_y=0.2, e_z=0.25$ (so the largest distortion is on the $(x,y)$-plane), and the width is $\Delta=0.2$. We evolve this initial data up to $t=31$ (approximately 10 light crossing times).  We will focus, separately, on quantities defined in the bulk of the numerical grid and quantities defined at the AdS boundary.

- Intro
. long run 
. completely asymmetric scalar field initial data
. convergence of bulk and boundary quantities
. preliminary conclusions about physical properties
%In this Section we present the details of a representative run starting from initial data with a completely asymmetric massless scalar scalar field. We will focus, separately, on quantities defined on the bulk of the numerical grid and quantities defined on the AdS boundary. We aim to show that these converge to their exact versions as we increase the grid resolution, and to draw some preliminary conclusions about physical properties of the solution, which will be analysed thoroughly in the future.



-Collapse and Ringdown .time-symmetry .scalar field profile (amplitude, eccentricities, delta) . scalar field snapshots .relative Kretschmann approach Schw-AdS with a certain AdS mass .collapse in 0 bounces in a time of order $amp^2$ (consistent with Bizon-Rostorowski prediction for spherical symmetry) .cascade to the UV modes for scalar field but amplitude of scalar field is decreasing, so instability might .grid function convergence .iresall convergence

- Boundary Stress-Energy Tensor .extrapolation technique .expression for components . analytic tracelessness .analytic conservation .convergence . tracelessness .conservation

\subsection{Collapse and ringdown}\label{sec:rescolring}

We describe here the evolution of bulk quantities (DECIDE IF WE INCLUDE THE RIEMANN CUBE-KRETSCHMANN COMPARISON). Figure \ref{fig:snapshotsscalarfield} shows four representative times of the scalar field profile $\bar{\varphi}$ on the equatorial plane $z=0$. Notice that, in all of these, $\bar{\varphi}=0$ at the AdS boundary as required by the Dirichlet boundary condition. At $t=0$, the asymmetry of the initial Gaussian profile is too small to be visible. At the beginning of evolution, we see that the scalar field starts spreading towards the AdS boundary but, before reaching it, it is attracted towards the centre and forms an apparent horizon at $t=0.3$. [CITE Bizon-Rostworowski] predicts a collapse time of order $1/A^2\simeq 3.3$ in spherical symmetry. Thus, as suggested by the results in [Bantilan-Figueras-Kunesch], asymmetry in absence of angular momentum seems to accelerate the collapse.
The asymmetry on the $(x,y)$-plane is clearly visible at $t=2.6$, where the scalar field is stretched along the $x$-direction. The elongation changes its direction multiple times during the evolution, as shown in the next two plots: it is along the $y$-axis at $t=5.0$ and again along the $x$-axis at $t=7.2$. Furthermore, we see that the largest part of the scalar field progressively disappears from the exterior part of the spacetime by entering the apparent horizon, hence the amplitude of the scalar field decreases. At later times (BE MORE SPECIFIC), the value of the scalar field becomes consistent with 0 up to solution error (estimated by comparing $\bar{\varphi}$ at different resolutions) and the spacetime settles down to a Schwarzschild-AdS black hole. (GOOD PLACE TO ATTACH THE RIEMANN CUBE-KRETSCHMANN COMPARISON)

\begin{figure}[h]
        \centering
        \includegraphics[width=5.0in,clip=true]{plots/bulkplots/L3p5/phi1/phi1_L3p5_snapshots_2by2.png}
\parbox{5.0in}{\caption{Snapshots of the scalar field profile $\bar{\varphi}$ on the $z=0$ slice in $(x,y)$ coordinates (L3p5 resolution, $N_x=N_y=N_z=487$). In each plot, the black square denotes the boundary of the numerical grid, i.e. $x=\pm 1$ and $y=\pm 1$. The external boundary of the coloured part is the AdS boundary $\rho=1$. The black ellipse is the elliptic approximation of the Apparent Horizon (AH) at each time step. The internal boundary of the coloured region is the excision surface: we excise points inside an ellipse whose semi-axes are the AH ellipse semi-axes reduced by a factor $1-ex_{rbuf}=0.65$.
        }\label{fig:snapshotsscalarfield}}
\end{figure}

\begin{figure}[h]
        \centering
        \includegraphics[width=4.0in,clip=true]{plots/timeseries/AdSmass/fullplotAdSmassL1L2L3res-cropped.pdf}
\parbox{5.0in}{\caption{Time series for AdS mass for different resolutions.
        }\label{fig:AdSmass-crop}}
\end{figure}

\begin{figure}[h]
        \centering
        \includegraphics[width=4.0in,clip=true]{plots/timeseries/L2norm_iresallconvergence/fullplotiresallconvergenceL1L2L3res-cropped.pdf}
\parbox{5.0in}{\caption{Time evolution for $L^2$ norm of convergence factor for independent residual of EFE at different resolutions.
        }\label{fig:L2norm_iresallconvergence-crop}}
\end{figure}

\begin{figure}[h]
        \centering
        \includegraphics[width=4.0in,clip=true]{plots/timeseries/L2norm_quasiset_trace/fullplotL2normtraceL2L3res-cropped.pdf}
\parbox{5.0in}{\caption{Time evolution for $L^2$ norm of trace of quasi-local energy-momentum tensor of the boundary CFT at different resolutions.
        }\label{fig:L2norm_quasisettrace-crop}}
\end{figure}




\subsection{Boundary stress-energy tensor}
\label{sec:resbouset}

In this section we look at snapshots at fixed $t$ on the boundary $S^2$ at $\rho=1$ for some of the quantities at the AdS boundary, calculated in Section~\ref{sec:bouset2}. All the plots are obtained from bulk quantities via third order extrapolation. The only exception is the $t=0$ plot of Figure \ref{fig:snapshotsscalarfield}: this is computed analytically from the initial distorted Gaussian profile \eref{eq:scaGaupro}, as extrapolation would give values smaller than the solution error of the function. 

Figure \ref{fig:snapshotsbdyphi} shows the scalar field of the boundary CFT, $\langle \bar{\varphi}\rangle_{CFT}$.
Unlike the $z=0$ slice snapshots of Figure \ref{fig:snapshotsscalarfield}, these sphere plots show the asymmetry in all three Cartesian direction. In fact, the asymmetry of the initial data is already visible at $t=0$, where the different values of eccentricities along the three Cartesian direction (largest along $x$ and smallest along $y$) are evident in this plot. At this time, the scalar field is overall very small, which is expected since the initial $\bar{\varphi}$, given by \eref{eq:scaGaupro}, is localised near $\rho=0$.
As noticed in Figure \ref{fig:snapshotsscalarfield}, the asymmetry changes direction during evolution, but interestingly it is always strongest along $x$ and weakest along $y$ or viceversa. Furthermore, a direct comparison with Figure \ref{fig:snapshotsscalarfield} shows that the features present at a certain $t$ at the boundary take approximately $\pi/2\simeq1.6$ to reach the interior of the bulk, i.e. about half light-crossing time, as expected. At later times, similarly to the bulk, the scalar field at the boundary becomes negligible as the spacetime settles down to Schwarzschild-AdS.

\begin{figure}[h]
        \centering
        \includegraphics[width=5.0in,clip=true]{plots/bdyplots/L3p5/bdyphi/sphereplots_bdyphi_L3p5_2by2.png}
\parbox{5.0in}{\caption{Snapshots of the scalar field of the boundary CFT, $\langle \bar{\varphi}\rangle_{CFT}$. The first snapshot is obtained analytically from the initial scalar field profile. The remaining three are obtained by third order extrapolation and subsequent smoothening via a low-pass filter with threshold frequency $\omega=0.05$ (L3p5 resolution, $N_x=487$).
        }\label{fig:snapshotsbdyphi}}
\end{figure}

Figure \ref{fig:snapshotsenergydensity} displays the energy density of the boundary CFT, $\epsilon_{CFT}$. At $t=0$ this is strongly asymmetric along the $x$-direction, as expected from the shape of the initial scalar field \eref{eq:scaGaupro}. After the formation of the apparent horizon, $\epsilon_{CFT}$ quickly becomes approximately uniform, as appropriate for the boundary energy density of Schwarzschild-AdS, and it remains essentially unchanged from $t=3.4$ onwards. Even though the boundary scalar field is non-vanishing in the early stages, it quickly becomes not large enough to affect the value of $\epsilon_{CFT}$.

\begin{figure}[h]
        \centering
        \includegraphics[width=5.0in,clip=true]{plots/bdyplots/L3p5/bdyenergydensity/sphereplots_bdyenergydensity_L3p5_2by2.png}
\parbox{5.0in}{\caption{Snapshots of energy density $\epsilon$ at the boundary, obtained by third order extrapolation and smoothened via a low-pass filter with threshold frequency $\omega=0.05$ (L3p5 resolution, $N_x=487$).
        }\label{fig:snapshotsenergydensity}}
\end{figure}

Figure \ref{fig:snapshotsenergydensityrelativetoSchw} displays the energy density of the boundary CFT relative to the Schwarzschild-AdS value, i.e. $\frac{\epsilon_{CFT}}{\epsilon_{CFT (S-AdS)}}-1$ where $\epsilon_{CFT (S-AdS)}=\frac{M}{4\pi}$ and for $M$ we use the AdS mass $M=0.410594$ of the last time shown, $t=5.6$. This quantity is initially significantly different from $\epsilon_{CFT (S-AdS)}$ but it quickly approaches its Schwarzschild-AdS analog and it remains essentially unchanged from $t=3.4$ onwards.

\begin{figure}[h]
        \centering
        \includegraphics[width=5.0in,clip=true]{plots/bdyplots/L3p5/bdyenergydensityrelativetoschw/sphereplots_bdyenergydensityrelativetoschw_L3p5_2by2.png}
\parbox{5.0in}{\caption{Snapshots of energy density relative to Schwarschild-AdS with the AdS mass of t=5.6, $M=0.410594$ i.e. $\frac{\epsilon_{CFT}}{\epsilon_{CFT (S-AdS)}}-1$, at the boundary (L3p5 resolution, $N_x=487$).
        }\label{fig:snapshotsenergydensityrelativetoSchw}}
\end{figure}

Figure \ref{fig:snapshotsbdyanisotropy} displays the anisotropy of the boundary CFT $\Delta p_{CFT}$. This is at all times consistent with the vanishing value of the Schwarzschild-AdS case.


\begin{figure}[h]
        \centering
        \includegraphics[width=5.0in,clip=true]{plots/bdyplots/L3p5/bdyanisotropy/sphereplots_bdyanisotropy_L3p5_2by2.png}
\parbox{5.0in}{\caption{Snapshots of anisotropy at the boundary (L3p5 resolution, $N_x=487$).
        }\label{fig:snapshotsbdyanisotropy}}
\end{figure}

%\begin{figure}%
%    \centering
%    \subfloat[x-dependency]{{\includegraphics[width=3.0in]{plots/bulkplots/L2/relkretsch/fullplotxcoordrelkretschL2res-cropped.pdf} }}%
%    \qquad
%    \subfloat[y-dependency]{{\includegraphics[width=3.0in]{plots/bulkplots/L2/relkretsch/fullplotycoordrelkretschL2res-cropped.pdf} }}%
%    \caption{Time evolution of Kretschmann: the Kretschmann scalar of the solution approaches the one for Schwarzschild-AdS}%
%    \label{fig:example}%
%\end{figure}




\section{Discussion}\label{sec:Discussion}

\textcolor{blue}{The calculation just outlined would be almost identical if we were to study 3+1 Cartesian evolution of asymptotically AdS spacetimes in any $d\geq4$ dimensions. In particular, the stable gauge found with this method would be the same up to a numerical factor. Even more interestingly, perhaps, a comparison between \eref{eqn:target_gauge_xyz},\eref{eqn:target_gauge_t} and the corresponding result in [cite arXiv:1706.04199 [hep-th] ] (see eq. (S10) ) clearly suggests a trend for the expression of the stable gauge in $D+1$ dimensions for any number $D\geq1$ of spatial dimensions. If this trend were confirmed, repeating the calculation above would not be necessary when changing the value of $D$.}

\ack
Simulations were run on the {\bf Apocrita} cluster at Queen Mary University of London.

\appendix
\section{Generalized Harmonic Formulation}
\label{sec:GHfor}

The generalized harmonic formulation of the Einstein equations is based on coordinates $x^\mu$ that each satisfies a wave equation $\Box x^{\mu}=H^\mu$ with source functions $H^\mu$.
As long as the constraint $0=C^\mu \equiv H^\mu-\Box x^\mu$ is satisfied, we can then write the trace-reversed Einstein equations in $d$ dimensions with cosmological constant $\Lambda$
\begin{equation}
0=R_{\mu\nu} - \frac{2\Lambda}{d-2} g_{\mu\nu} - 8\pi\left( T_{\mu\nu} - \frac{1}{d-2} T g_{\mu\nu} \right)
\end{equation}
as
\begin{eqnarray}
0
&=& R_{\mu\nu} - \nabla_{(\mu} C_{\nu)} - \frac{2\Lambda}{d-2} g_{\mu\nu} - 8\pi\left( T_{\mu\nu} - \frac{1}{d-2} {T^\alpha}_\alpha g_{\mu\nu} \right) \nonumber \\
&=& R_{\mu\nu} - \nabla_{(\mu} H_{\nu)} + \nabla_{(\mu} \Box{x}_{\nu)} - \frac{2\Lambda}{d-2} g_{\mu\nu} - 8\pi\left( T_{\mu\nu} - \frac{1}{d-2} {T^\alpha}_\alpha g_{\mu\nu} \right) \nonumber \\
&=& -\frac{1}{2} g^{\alpha\beta} g_{\mu\nu,\alpha\beta} - g^{\alpha\beta}{}_{,(\mu}g_{\nu)\alpha,\beta} - H_{(\mu,\nu)} + H_\alpha \Gamma^\alpha{}_{\mu\nu} \nonumber \\
&&- \Gamma^\alpha{}_{\mu\beta}\Gamma^\beta{}_{\alpha\nu} - \frac{2\Lambda}{d-2} g_{\mu\nu} - 8\pi\left( T_{\mu\nu} - \frac{1}{d-2} {T^\alpha}_\alpha g_{\mu\nu} \right) \nonumber,
\end{eqnarray}
where the choice of $H_\mu = g_{\mu\nu} H^\nu$ constitutes a gauge choice ($\Gamma^\mu{}_{\nu\rho}$ are the Christoffel symbols of the metric in the chosen set of coordinates and $T_{\mu\nu}$ is the stress-energy tensor of any matter field coupled to the metric). 
To suppress constraint-violating solutions that do not satisfy $C_\mu=0$, we supplement with constraint-damping terms as introduced in~\cite{Gundlach:2005eh} and obtain our final form of the Einstein equations
\begin{eqnarray}\label{eqn:efe_gh_modified}
&-& \frac{1}{2} g^{\alpha \beta} g_{\mu \nu, \alpha \beta} - 
{g^{\alpha \beta}}_{,(\mu} g_{\nu) \alpha, \beta} - H_{(\mu, \nu)} + H_\alpha {\Gamma^\alpha}_{\mu \nu} \nonumber \\
&-& {\Gamma^\alpha}_{\beta \mu} {\Gamma^\beta}_{\alpha \nu} - \kappa \left( 2 n_{(\mu} C_{\nu)} - (1+P) g_{\mu \nu} n^\alpha 
C_\alpha \right) \nonumber \\
&=&   \frac{2}{d-2} \Lambda g_{\mu \nu} + 8\pi \left( T_{\mu \nu} - 
\frac{1}{d-2} {T^\alpha}_\alpha g_{\mu \nu} \right).
\end{eqnarray}

See, for example, CITE [arXiv:gr-qc/0407110] , [arXiv:1201.2132v3 [hep-th]]  for more details.

In this work, we are interested in the case where matter fields are given by a single massless real scalar field $\varphi$, hence the stress-energy tensor reads
\begin{equation}
T_{\mu\nu}=\partial_\alpha \varphi \partial_\beta \varphi - g_{\alpha\beta} \frac{1}{2} g^{\gamma\delta} \partial_{\gamma} \varphi \partial_{\delta} \varphi,
\end{equation}
and $\varphi$ is coupled to $g_{\mu\nu}$ through the Klein-Gordon equation \eref{eqn:eoms2}, which we write here for completeness in terms of partial derivatives w.r.t. the chosen set of coordinates:
\begin{equation}\label{eqn:eoms2cart}
g^{\mu\nu} \partial_{\mu} \partial_{\nu} \varphi -g^{\mu\nu} \Gamma^{\rho}{}_{\nu\mu}\partial_\rho\varphi= 0.
\end{equation}

\section{Evolution Variables in Spherical Coordinates}\label{sec:sphevvarboucon}

Here we apply the prescription of Section~\ref{subsec:cartevvarboucon} to the case of asymptotically AdS spacetimes in spherical coordinates, in order to construct the spherical coordinate version of the Cartesian evolution variables $(\bar{g}_{\mu\nu},\bar{\varphi},\bar{H}_\mu)$ and how these two sets of variables in different coordinates are related to each other.

We remind the reader that new evolution variables are defined in order to apply the boundary conditions found in Section~\ref{subsec:asyAdS} as simple Dirichlet conditions at the AdS boundary $\rho=1$. In the same way as the Cartesian coordinate case, the metric evolution variables in spherical coordinates $\bar{g}_{\alpha\beta}$ are defined by (i) considering the deviation from pure AdS tensor $h_{\alpha\beta}=g_{\alpha\beta}-\hat{g}_{\alpha\beta}$ in spherical coordinates, (ii) stripping $h_{\alpha\beta}$ of as many factors of $(1-\rho^2)$ as needed so that they fall off linearly in $(1-\rho)$ near the AdS boundary at $\rho=1$.

The boundary conditions on $h_{\alpha\beta}$ \eref{eq:sphbounconh} tell us that
\begin{eqnarray}\label{eq:gbarsph}
\bar{g}_{\rho\alpha}=\frac{h_{\rho\alpha} }{1-\rho^2}\;\; \textrm{ if $\alpha\neq\rho$}, \\ \nonumber
\bar{g}_{\alpha\beta}=h_{\alpha\beta}  \;\; \textrm{ otherwise}.
\end{eqnarray}

Despite the notation, we emphasize that $\bar{g}_{\alpha\beta}$ and $\bar{g}_{\mu\nu}$ are not in general components of the same tensor (as it should be clear from their definition), therefore the usual transformation between tensor components in different sets of coordinates cannot be applied. The correct transformation can be easily deduced from \eref{eq:gbarsph} and \eref{eq:gbarcart} remembering that $h$ is indeed a tensor: 
\begin{eqnarray}\label{eq:cartosph}
\bar{g}_{\rho\alpha}&=&\frac{1}{(1-\rho^2)}\frac{\partial x^\mu}{\partial \rho}\frac{\partial x^\nu}{\partial x^\alpha}\bar{g}_{\mu\nu}\;\; \textrm{ if $\alpha\neq\rho$}, \\ \nonumber
\bar{g}_{\alpha\beta}&=&\frac{\partial x^\mu}{\partial x^\alpha}\frac{\partial x^\nu}{\partial x^\beta}\bar{g}_{\mu\nu}\;\; \textrm{ otherwise}.
\end{eqnarray}

Similarly, the boundary conditions on the scalar field \eref{eq:sphbounconphi} suggest that we use the evolution variable
\begin{equation}
\bar{\varphi}=\frac{\varphi }{(1-\rho^2)^2}.
\end{equation}
which the same as the one in Cartesian coordinates, as expected for a scalar field.

Finally, the boundary conditions \eref{eq:sphbouncondsoufunc} on $H_\alpha$ suggest the use of the evolution variables
\begin{eqnarray}
 \bar{H}_\alpha=\frac{H_\alpha-\hat{H}_\alpha}{(1-\rho^2)^2 } \;\; \textrm{ if $\alpha\neq\rho$,} \\ \nonumber
 \bar{H}_\rho=\frac{H_\rho-\hat{H}_\rho}{1-\rho^2 }
 \end{eqnarray}
in spherical coordinates.

Neither $H_\alpha,\hat{H}_\alpha,\bar{H}_\alpha$ nor $H_\mu,\hat{H}_\mu,\bar{H}_\mu$ are components of the same tensor, so there is no direct transformation from one set to the other. The two triplets of quantities can only be obtained from the definition of source functions in terms of the full metric $g$ in the appropriate set of coordinates, e.g. \eref{eq:sphbouncondsoufunc} in spherical coordinates.

\section{Gauge Choice in the Bulk}
\label{sec:GCbulk}

In Section~\ref{sec:gauge_choice} we discussed the gauge choice of source functions near the boundary that provides stable evolution. However the choice in the rest of the computational domain is still arbitrary. In this Section we make this choice explicit.

\section{Convergence Tests in the Bulk}\label{sec:convbulk}

Here we display a pair of numerical tests that demonstrate convergence trends for a representative simulation with no mesh refinement.
We do this for [pick simulation] with initial data parameters [display parameters]

Let us denote the value of the field $f$ at the point $(t,x,y,z)$ in a simulation with mesh spacing $\Delta$ by $f_\Delta(t,x,y,z)$.
To show that the solution $f_\Delta(t,x,y,z)$ converges to a function $f(t,x,y,z)$ in the continuum limit $\Delta\rightarrow0$, we compute the rate of convergence $Q(t,x,y,z)$ at each point of interest
\begin{equation}\label{eq:qconv}
Q(t,x,y,z)=\frac{1}{\ln(3/2)}\ln\left( \frac{f_{9h/4}(t,x,y,z)-f_{3h/2}(t,x,y,z)}{f_{3h/2}(t,x,y,z)-f_{h}(t,x,y,z)} \right).
\end{equation}
We use second-order accurate finite difference stencils, and there is a factor of 3/2 between successive resolutions.
Thus, we expect $Q$ to asymptote to $Q=2$ in the limit $\Delta\rightarrow0$.

To show that the solution is converging to a solution of the Einstein equations, we compute an independent residual. 
This is obtained by taking the numerical solution and substituting it into a discretized version of
the Einstein equations, a component of which we denote $\Phi_\Delta$. 
The independent residual should be purely numerical truncation error, so we can compute a convergence factor for it by using only two resolutions
\begin{equation}\label{eq:qires}
Q_{EFE}(t,x,y,z)=\frac{1}{\ln(3/2)}\ln\left( \frac{\Phi_{3h/2}(t,x,y,z)}{\Phi_{h}(t,x,y,z)} \right).
\end{equation}
Again, with second-order accurate finite difference stencils and with a factor of 3/2 between successive resolutions, we expect $Q$ to approach $Q=2$ as $\Delta\rightarrow0$.

\section{Extrapolation Technique and Convergence at AdS Boundary}\label{sec:extrapconvbdy}

As explained in Section~\ref{sec:bouset2}, given a Cartesian grid with spacing $\Delta$, since the AdS boundary generally does not lie on points of the grid, we can only obtain the approximated value $f^{bdy}_{\Delta}$ of any boundary quantity $f$ through extrapolation, using the numerical values of $f$ on grid points near the boundary, which we denote by $f_\Delta$. In particular, we use first order extrapolation. In this Section we describe how grid points are chosen for extrapolation and we show convergence for the extrapolation scheme by computing the factor \eref{eq:qconv} for $f^{bdy}_{\Delta}$ at boundary points, which is expected as a consequence of bulk convergence (see~\ref{sec:convbulk}) at the chosen points.

If we wish to test convergence of $f^{bdy}_{\Delta}$ in the continuum limit $\Delta\rightarrow0$, there are two crucial criteria for the choice of grid points that are \emph{suitable} for extrapolation. We will now explain them with the help of Fig.~\ref{fig:lego_circle}, which shows an example of the result of the application of these criteria in the first quadrant of a $z=const.$ surface of the grid for first order extrapolation.
\begin{enumerate}
 \item Consider the value of $f^{bdy}_{\Delta}$ at any boundary point $p_{bdy}$. If we obtain this by $1^{st}-$order extrapolation, $f^{bdy}_{\Delta}(p_{bdy})$ is a combination of the values $f_\Delta(p_1),f_\Delta(p_2)$ (where $p_1,p_2$ are the bulk points used for extrapolation) multiplied by factors that depend on the coordinates of $p_{bdy},p_1,p_2$. Therefore, we can write  $f^{bdy}_{\Delta}(p_{bdy})=f(p_{bdy})+c_{extr}(p_{bdy},p_1,p_2)+c_\Delta(p_1,p_2)\Delta^2$ where the first term is the true value of $f$ at $p_{bdy}$, the second term is the error due to the extrapolation approximation and the third term is the error coming from the numerical error in $f_\Delta$. From this we immediately see that the convergence factor \eref{eq:qconv} at point $p_{bdy}$ can be expected to asymptote to 2 as $\Delta\rightarrow0$ only if the points $p_1,p_2$ are the same for all 3 resolutions involved.
 
 
\item For each resolution, grid points with $\rho\geq 1-\Delta/2$ (points outside the dotted line in Fig.~\ref{fig:lego_circle}) are excised due to the fact that some quantities diverge at $\rho=1$, the value of functions at points with $1-3\Delta/2\leq \rho < 1-\Delta/2$ (points between the dotted line and the dashed line in Fig.~\ref{fig:lego_circle}) are set by using forward/backward stencils, and the value of functions at points with $1-5\Delta/2\leq \rho < 1-3\Delta/2$ (between the dashed line and the continuous blue line in Fig.~\ref{fig:lego_circle}) are set by using centred stencils which, however, use neighbouring points set by forward/backward stencils. For all these reasons, we can only expect convergence if we pick grid points with  $\rho<1-5\Delta/2$ (inside the continuous blue line in Fig.~\ref{fig:lego_circle}).
 \end{enumerate}
Furthermore, among the points satisfying (i) and (ii), we want the ones that are closest to the boundary in order for extrapolation to provide a more accurate approximation.

%If we wish to test convergence of $f^{bdy}_{\Delta}$ in the continuum limit $\Delta\rightarrow0$, there are two crucial criteria for the choice of grid points that are \emph{suitable} for extrapolation. We will now explain them with the help of Fig.~\ref{fig:lego_circle}, which shows an example of the result of the application of these criteria in the first quadrant of a $z=const.$ surface of the grid for first order extrapolation.
%\begin{enumerate}
% \item Consider the value of $f^{bdy}_{\Delta}$ at any boundary point $p_{bdy}$. If we obtain this by $n^{th}-$order extrapolation, $f^{bdy}_{\Delta}(p_{bdy})$ is a combination of the values $f_\Delta(p_1),f_\Delta(p_2),\dots,f_\Delta(p_{n+1})$ (where $p_1,p_2,\dots,p_{n+1}$ are the bulk points used for extrapolation) multiplied by factors that depend on the coordinates of $p_{bdy},p_1,\dots,p_{n_{\Delta}+1}$. Therefore, we can write  $f^{bdy}_{\Delta}(p_{bdy})=f(p_{bdy})+c_{extr}(p_{bdy},p_1,p_2,\dots,p_{n+1})+c_\Delta(p_1,p_2,\dots,p_{n+1})\Delta^2$ where the first term is the true value of $f$ at $p_{bdy}$, the second term is the error due to the extrapolation approximation and the third term is the error coming from the numerical error in $f_\Delta$. From this we immediately see that the convergence factor \eref{eq:qconv} at point $p_{bdy}$ can be expected to asymptote to 2 as $\Delta\rightarrow0$ only if the points $p_1,p_2,\dots,p_{n+1}$ are the same for all 3 resolutions involved.
 
 %\item For each resolution, grid points with $\rho\geq 1-\Delta/2$ (points outside the dotted line in Fig.~\ref{fig:lego_circle}) are excised due to the fact that some quantities diverge at $\rho=1$, the value of functions at points with $1-3\Delta/2\leq \rho < 1-\Delta/2$ (points between the dotted line and the dashed line in Fig.~\ref{fig:lego_circle}) are set by using forward/backward stencils, and the value of functions at points with $1-5\Delta/2\leq \rho < 1-3\Delta/2$ (between the dashed line and the continuous blue line in Fig.~\ref{fig:lego_circle}) are set by using centred stencils which, however, use neighbouring points set by forward/backward stencils. For all these reasons, we can only expect convergence if we pick grid points with  $\rho<1-5\Delta/2$ (inside the continuous blue line in Fig.~\ref{fig:lego_circle}).
 %\end{enumerate}
%Furthermore, among the points satisfying (i) and (ii), we want the ones that are closest to the boundary in order for extrapolation to provide a more accurate approximation.

\begin{figure}[h]
        \centering
        \includegraphics[width=6.0in,clip=true]{plots/lego_circle/Lego_circle.pdf}
\parbox{5.0in}{\caption{Visual description of first order extrapolation technique in the first quadrant of a $z=const.$ surface for the lowest resolution, with grid spacing $\Delta$, among the three involved in a convergence test ($const.$ is chosen as the $z-$coordinate value of one of the points common to all three resolutions). We show a typical example in which common points are 4 grid points away from each other along each direction.
        }\label{fig:lego_circle}}
\end{figure}
 
Once suitable points have been found (the green dots in Fig.~\ref{fig:lego_circle}), the extrapolation technique proceeds as follows:
 \begin{itemize}
 \item consider the first suitable point, $p_1$, and identify the coordinate with the largest value, e.g. $x$, and its sign, e.g. $x>0$. If two coordinates have the same value, then we pick $x$ over $y$ or $z$ and $y$ over $z$.
 \item take the next point, $p_2$, used for $1^{st}-$order extrapolation along the identified axis ($x$ in our example) in the direction of the bulk (decreasing $x$ in the example), making sure that they also satisfy (i) ((ii) is trivially satisfied by construction). Each $p_2$ is represented in yellow in Fig.~\ref{fig:lego_circle}.
 \item use $1^{st}-$ order extrapolation on $f_\Delta(p_1),f_\Delta(p_{2})$ to determine the value of $f^{bdy}_{\Delta}(p_{bdy})$ where $p_{bdy}$ is the boundary point along the identified axis in the direction of the boundary (one of the red dots in Fig. ~\ref{fig:lego_circle}). In our example, $p_{bdy}$ is the point with coordinates $(x,y,z)=(\sqrt{1-y(p_1)^2-z(p_1)^2},y(p_1),z(p_1))$.
 \item repeat the previous steps until the last suitable point.
 \end{itemize}

CONVERGENCE PLOTS

This extrapolation scheme is validated by Figure [CONVERGENCE], which shows second-order convergence at the boundary for CHOSEN FUNCTION with this extrapolation scheme, as expected from having second-order convergence in the bulk (see ~\ref{sec:convbulk}) at all points used for extrapolation.

%Notice, from the discussion in (i), that we do not expect to converge to the true value $f(p_{bdy})$ of a boundary quantity in the continuum limit $\Delta\rightarrow0$, but rather to $f(p_{bdy})+c_{extr}(p_{bdy},p_1,p_2,\dots,p_{n+1})$.
%For this reason, the convergence test \eref{eq:qires} cannot be performed at the boundary for functions with true value 0 (such as $\langle trT \rangle_{CFT}$), because their extrapolated value is not just the term linear in $\Delta^2$ but it also includes the extrapolation error $c_{extr}$.

Notice, from the discussion in (i), that we do not expect to converge to the true value $f(p_{bdy})$ of a boundary quantity in the continuum limit $\Delta\rightarrow0$, but rather to its approximation $f(p_{bdy})+c_{extr}(p_{bdy},p_1,p_2)$.
For this reason, the convergence test \eref{eq:qires} cannot be performed at the boundary for functions with true value 0 (such as $\langle trT \rangle_{CFT}$), because their extrapolated value is not just the term linear in $\Delta^2$ but it also includes the extrapolation error $c_{extr}$.




%-------------------------------------------------------
% Bibliography
%-------------------------------------------------------
\section*{References}
\bibliographystyle{iopart-num}
\bibliography{3p1}



\end{document}

