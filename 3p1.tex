\documentclass[12pt]{iopart}
%Uncomment next line if AMS fonts required
%\usepackage{iopams}  
\setlength\parindent{0pt}
\setlength{\parskip}{13pt} %found this single-line pt value using \the\baselineskip
\begin{document}

\title[]{Cauchy Evolution of Asymptotically AdS Spacetimes with No Symmetries}

\author{Hans Bantilan$^1$, Pau Figueras$^1$, Lorenzo Rossi$^1$}
\address{$^1$ School of Mathematical Sciences, Queen Mary University of
  London, \\ Mile End Road, London E1 4NS, United Kingdom}
\ead{l.rossi@qmul.ac.uk}

\begin{abstract}
We present the first proof-of-principle Cauchy evolutions of asymptotically AdS spacetimes with no symmetries.
We use a numerical scheme that is based on the generalized harmonic form of the Einstein equations expressed in Cartesian coordinates.
For this first study, we perform 3+1 simulations of four dimensional spacetimes with a negative cosmological constant, using initial data sourced by a massless scalar field.
We demonstrate stable and convergent Cauchy evolution of horizonless data, as well as data that undergoes gravitational collapse.
We explicitly write down the gauge choice that achieves this in terms of a specific choice of generalized harmonic source functions.
These preliminary results are a direct precursor to important specialized studies of AdS dynamics with no symmetry assumptions, most notably of gravitational collapse, and of the superradiant instability.
\end{abstract}


\noindent{Keywords}: AdS, generalized harmonic, 3+1, superradiance, gravitational collapse

% Uncomment for Submitted to journal title message
%\submitto{\JPA}
%
% Uncomment if a separate title page is required
%\maketitle
% 
% For two-column output uncomment the next line and choose [10pt] rather than [12pt] in the \documentclass declaration
%\ioptwocol




\section{Introduction}

In an unprecedented way, the simulation of asymptotically anti-de Sitter (AdS) spacetimes has opened up the field of numerical relativity to the study of phenomena in areas beyond General Relativity (GR). 
At the heart of this push to understand AdS is a deep mathematical connection between gravity in AdS to certain conformal field theories (CFT), now known as the AdS/CFT correspondence~\cite{Maldacena:1997re,Gubser:1998bc,Witten:1998qj}. 
Through this connection, the study of AdS spacetimes has become immediately relevant to fundamental questions in many areas in physics, such as fluid dynamics [cite], relativistic heavy ion collisions [cite], and superconductivity [cite].
See, for example, [cite] for excellent reviews. 

The reason the study of AdS is crucial for our understanding of these phenomena is that AdS/CFT provides an important --and in most cases the only-- window into the real-time dynamics of quantum field theories far from equilibrium. 
The dynamical far-from-equilibrium regime is precisely the one that is least explored and understood, and the one that has the best chance of making contact with experiment.
Within GR, the dynamics of AdS provides a detailed look at how the Einstein field equations behave in the fully non-linear regime and with non-trivial boundary conditions.
This has led to important progress in our understanding fundamental phenomena in gravity, such as gravitational collapse, black hole collisions, and superradiance. 

Our current understanding of AdS, however, remains limited. 
This is due to several reasons.
First, evolution in AdS is notoriously hard, in part because it is an initial boundary value problem whose systematic study is still in its infancy. 
Cauchy evolution in AdS requires data to be prescribed not only at an initial space-like hypersurface, but also at spatial and null infinity which constitute the time-like boundary of an asymptotically AdS spacetime.
Second, the most interesting phenomena involve spacetimes that have very little or no symmetry, making these evolutions beyond the reach of most numerical codes. 
Third, for many of these phenomena, there is a large separation of scales that must be adequately resolved to capture the correct physics.

The main purpose of this article is to present the first proof-of-principle Cauchy evolution of asymptotically AdS spacetimes that has been achieved with no symmetry assumptions.
The results presented here are based on a code that solves the Einstein equations for asymptotically AdS spacetimes with a generalized harmonic evolution scheme, with adaptive mesh refinement capabilities. 
This code is the next step in an ongoing program initiated in \cite{Bantilan:2017kok} that uses Cartesian coordinates in global AdS, now enabled with a breakthrough set of generalized harmonic source functions that is crucial in successfully stabilizing evolution in AdS with no symmetries.

This article is organized as follows. 
In Section~\ref{sec:numerical_scheme} we describe the generalized harmonic scheme that we use in our simulations, and details the gauge choice in Section~\ref{sec:gauge} in terms of a specific choice of generalized harmonic source functions that respresents the crucial result that stabilizes these evolutions with no symmetry.
Section~\ref{sec:results} contains preliminary results from this new code, and we conclude with a discussion in Section~\ref{sec:Discussion}.



\section{Numerical Scheme}\label{sec:numerical_scheme}

\section{Gauge}\label{sec:gauge}

\section{Results}\label{sec:results}

\section{Discussion}\label{sec:Discussion}

\cite{Bantilan:2012vu}







%-------------------------------------------------------
% Bibliography
%-------------------------------------------------------
\section*{References}
\bibliographystyle{iopart-num}
\bibliography{3p1}



\end{document}

